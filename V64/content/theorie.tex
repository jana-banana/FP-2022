\section{Zielsetzung}
\label{sec:Ziel} 
\noindent
In diesem Versuch wird das Sagnac-Interferometer als moderne Apparatur der Interferometrie vorgestellt. Die grundlegende Funktionsweise soll durch die Justage des Interferometers 
und die Konstrastmessung vermittelt werden. Anschließend wird mithilfe des Interferometers der Brechungindex von Glas sowie des Gasgemisches Luft bestimmt. 

\section{Theorie}
\label{sec:Theorie}


\subsection{Kohärenz}

\noindent Interferenz kann bei der Überlagerung von elektromagnetischen Wellen auftreten in Form der konstruktiven und destruktiven Interferenz. Konstruktive Interferenz bezeichnet dabei das
Addieren der Wellenmaxima, destruktive Interferenz das Aufheben der Wellen, sodass keine Intensität mehr gemessen wird. Damit zwei elektromagnetische Wellen miteinander interferieren können, 
müssen sie kohärent sein, sie müssen also eine feste Phasendifferenz und die gleiche Wellenlänge haben. Dies ist häufig nur über eine gewisse Zeitdauer der Fall, diese Zeit wird als 
Kohärenzzeit bezeichnet. Desweiteren wird in räumliche und zeitliche Kohärenz unterschieden, räumliche Kohärenz beschreibt eine feste Phasendifferenz bezüglich der Raumachse, zeitliche 
dementsprechend eine feste Phasendifferenz bezüglich der Zeitachse. 


\subsection{Polarisation}

\noindent Die Polarisation beschreibt die Auslenkungsrichtungs der Lichtwelle im Bezug zur Ausbreitungsrichtung. Es gibt unpolarisiertes Licht, linear polarisiertes und ellipstisch/
zirkular polarisiertes Licht. Beim letzteren dreht sich die Auslenkungsrichtung um die Bewegungsrichtung. Lineare Polarisation bedeutet, dass die Auslenkung der Welle senkrecht zur 
Bewegungsrichtung in einer Ebene stattfindet. Lichtwellen müssen gleich polarisiert sein, damit sie interferieren können. 


\subsection{Kontrast}

\noindent Die Qualität eines Interferometers kann mit dem Kontrast $K$ beschrieben werden:
\begin{equation}
        K = \frac{I_{\text{max}} - I_{\text{min}}}{I_{\text{max}} + I_{\text{min}}}
        \label{eqn:kontrast}
\end{equation}
Die Intensität $I$ ergibt sich durch zeitliche Mittelung des elektrischen Feldes über eine Periode, $I_{\text{max/min}}$  bezeichnet die Intensität der Interferenzmaxima/-minima.
Der Kontrast $K(\phi)$ ist im allgemeinen von dem Polarisationswinkel $\phi$ des ersten Polarisators abhängig. 

\noindent Die sich überlagernden Lichtwellen können definiert werden als 
\begin{align*}
    \vec{E}_1 &= \vec{E}_0 \cos(\phi)\cos(\omega t) & \vec{E}_2 &= \vec{E}_0 \sin(\phi)\cos(\omega t + \delta),
\end{align*} 
wobei $\omega$ die Kreisfrequenz der Welle und $\delta$ die Phasenverschiebung bezeichnet. Die Intensität berechnet sich dann mit der zeitlichen Mittelung über eine Periode:
\begin{equation*}
    I  \propto \langle \left( \vec{E}_1 + \vec{E}_2 \right)^2 \rangle = \langle E_1^2 + 2 E_1 E_2 + E_2^2 \rangle, 
\end{equation*} 
hierbei bezeichnet $\langle ... \rangle = \frac{1}{T} \int_t^{t+T} ... \, \text{d}t$. 
Die einzelnen Terme ergeben sich somit zu:
\begin{align*}
    \langle E_1^2 \rangle &= E_0^2 \cos^2(\phi) \langle \cos^2(\omega t) \rangle = \frac{1}{2} E_0^2 \cos^2(\phi) \\
    \langle E_2^2 \rangle &= E_0^2 \sin^2(\phi) \langle \cos^2(\omega t + \delta) \rangle = \frac{1}{2} E_0^2 \sin^2(\phi) \\
    \langle 2 E_1 E_2 \rangle &= 2 E_0^2 \sin(\phi)\cos(\phi) \langle \cos(\omega t) \cos(\omega t + \delta) \rangle = E_0^2 \sin(\phi) \cos(\phi) \cos(\delta) 
\end{align*}
Für die Intensität folgt dann:
\begin{align*}
    I &\propto  \frac{1}{2} E_0^2 \cos^2(\phi) + \frac{1}{2} E_0^2 \sin^2(\phi) + E_0^2 \sin(\phi) \cos(\phi) \cos(\delta)\\
    & \propto \frac{1}{2} E_0^2 \left( 1 + 2 \sin(\phi) \cos(\phi) \cos(\delta) \right)
\end{align*}
Für Interferenzmaxima $\left( \delta = 2n\pi, n \in \mathbb{N} \right)$ und -minima $\left(\delta = (2n+1)\pi, n \in \mathbb{N} \right)$ ergibt sich schließlich der Zusammenhang 
\begin{equation*}
    I_{\text{max/min}} \propto I_{\text{ges}} \left( 1 \pm 2 \cos(\phi)\sin(\phi)\right) . 
\end{equation*}

\noindent Der Kontrast hat somit die folgende Abhängigkeit vom Polarisationswinkel $\phi$ :
\begin{equation}
    V \propto 2 I_{\text{ges}} \sin(\phi)\cos(\phi) \label{eqn:kontrastphi}
\end{equation}


\subsection{Brechungsindexbestimmung von Glas}

\noindent Eine typische Anwendung der Interferometrie ist das Bestimmen von Brechungsindices verschiedener Materialien. Da die Lichtgeschwindigkeit in Materialien gegeben ist durch
$ v = \frac{c}{n}$ verändert sich der Wellenvektor gemäß $ k = \frac{2 \pi}{\lambda_0} \cdot n$, wobei $\lambda_0$ die Vakuumwellenlänge ist. \\
Beim Durchlaufen einer planparallelen Platte wird ein Lichtstrahl an beiden Grenzflächen mit dem gleichen Winkel entgegengesetzt gebrochen, sodass die Bewegungsrichtung nicht
verändert ist, aber der Strahl leicht versetzt ist. Da die Brechung vom Drehwinkel $\Theta$ abhängig ist, ergibt sich ein Ausdruck $\delta(\Theta)$ für den Fall, dass ein Strahl durch 
eine Glasplatte der Dicke $D$ läuft:
\begin{equation*}
    \delta = \frac{2 \pi D}{\lambda_0} \left[ \frac{n-1}{2n} \Theta^2 + \symcal{O}(\Theta ^4)\right]
\end{equation*}
In diesem Versuch wird mit zwei Glasplatten gearbeitet, die um $ \pm \Theta_0$ mit $\Theta_0 = \SI{10}{\degree}$ gegeneinander verschoben sind. Daher ergibt 
sich für die Phasenverschiebung 
\begin{equation*}
    \delta(\Theta) = \frac{2 \pi}{\lambda_0} D \frac{n-1}{2n} \left[ (\Theta + \Theta_0)^2 - (\Theta - \Theta_0)^2 \right]. 
\end{equation*}
Mit $ M = \frac{\delta}{2 \pi} $ ergibt sich für den Brechungsindex:
\begin{equation}
    n(\Theta) = \frac{1}{1 - \frac{M \lambda_0}{2 D \Theta \Theta_0}} \label{eqn:n_Glas}
\end{equation}


\subsection{Brechungsindexbestimmung von Gas}

\noindent Durchläuft der Lichtstrahl nun eine Kammer der Länge $L$, welche mit einem Gas des Brechungsindex $n$ gefüllt ist, so erfährt er eine 
Phasenverschiebung von 
\begin{equation*}
    \delta = \frac{2 \pi}{\lambda_0} \left(n-1\right) \cdot L.
\end{equation*}
Die Anzahl der Maxima oder der Minima $M$ steht durch 
\begin{equation*}
    M = \frac{\delta}{2 \pi}
\end{equation*}
im Zusammenhang mit der Phasenverschiebung, sodass bei gegebener Anzahl der Maxima oder Minima der Brechungsindex wie folgt gerechnet werden kann:
\begin{equation}
    n = \frac{\lambda_0\cdot M}{L} + 1
    \label{eqn:n_Luft}
\end{equation}
\noindent Über das Lorentz-Lorenz-Gesetz ist der Brechungsindex abhängig von dem Druck $p$ und der Temperatur $T$ des Gases. Durch eine Taylorentwicklung um $n \approx 1$ kann auf eine 
lineare Abhängigkeit genähert werden:
\begin{align*}
    \frac{A \cdot p}{R \cdot T} &= \frac{n^2 - 1}{n^2 + 2} \\
    &\approx \frac{2}{3}\left( n - 1 \right) 
\end{align*}
Der Brechungsindex kann somit über 
\begin{equation}
    n \approx \frac{3}{2} \frac{A \cdot p}{R \cdot T} + 1
    \label{eqn:Lorentz_Lorenz}
\end{equation}
bestimmt werden. $A$ bezeichnet hier die Refraktivität und $R$ die allgemeine Gaskonstante.