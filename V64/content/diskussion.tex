\section{Diskussion}
\label{sec:Diskussion}
Die aufgenommenen Messwerte und die ermittelten Ergebnisse entsprechen den Erwartungen.
Die bestimmten Kontraste folgen der erwarteten Verteilung und es sind keine größeren Abweichungen erkennbar.
Die Extrema sind auch wie erwartet bei Vielfachen von $45°$ zu finden.

\noindent
Für den Brechungsindex von Glas wurde $n_\text{Glas} = 1.0001079364890089$ bestimmt.
Der Theoriewert liegt bei $n_\text{Glas, Theo} = 1.45$ \cite{Brechungsindex}.
Der ermittelte Wert hat somit eine Abweichung von 31\%.

\noindent
Bei der Bestimmung des Brechungsindex für Luft wurde die Messung zwei Mal ohne einer Haube und zwei Mal mit einer Haube durchgeführt.
Die Haube minimiert Störungen durch Luftschwankungen.
Der ermittelte Brechungsindex beim ersten Durchgang beträgt $n_\text{Luft,ohne} = 1.0001703 \pm 0.0000017 $ und beim zweiten Durchgang mit der Haube $n_\text{Luft,ohne} =  1.0001622 \pm 0.0000016 $.
Der Theoriewert liegt für Luft bei $n_\text{Luft, Theo} = 1.000 292$ \cite{Brechungsindex}.
Die Abweichung von diesem Theoriewert liegt damit bei 0.01218\% bzw. 0.01298 \%.
Beide ermittelten Werte liegen Nahe dem Theoriewert.
Die Haube bewirkte also einen sehr kleinen Unterschied.
Denoch zeigt der Vergleich, dass Störungen durch Luftschwankungen einen Unterschied bei der Ermittlung machen.
Da der Brechungsindex von Luft im Allgemeinen sehr nah an 1 liegt, können diese Störungen einen relativ starken Einfluss haben.