\section{Diskussion}
\label{sec:Diskussion}
Die aufgenommenen Messwerte und die ermittelten Ergebnisse entsprechen den Erwartungen.
Eine Übersicht der ermittelten und der theoretische Werte, sowie die jeweilige Abweichung ist in \autoref{tab:ergebnisübersicht} zu finden.
Die bestimmten Kontraste folgen der erwarteten Verteilung und es sind keine größeren Abweichungen erkennbar.
Die Extrema sind auch wie erwartet bei Vielfachen von $45°$ zu finden.

\noindent
Für den Brechungsindex von Glas wurde $n_\text{Glas} = 1,55044793$ bestimmt.
Der Theoriewert liegt bei $n_\text{Glas, Theo} = 1,45$ \cite{Brechungsindex}.
Der ermittelte Wert hat somit eine Abweichung von $\SI{6.93}{\percent}$.
Ein Grund für diese etwas höhrere Abweichung kann in der Versuchsdurchführung liegen.
Hier wurden die Intensitätsmaxima bzw. -minima nur mit einer statt zwei Dioden aufgenommen.

\noindent
Bei der Bestimmung des Brechungsindex für Luft wurde die Messung zwei Mal ohne einer Haube und zwei Mal mit einer Haube durchgeführt.
Die Haube minimiert Störungen durch Luftdruckschwankungen.
Der ermittelte Brechungsindex beim ersten Durchgang beträgt $n_\text{Luft,ohne} = 1,000331 \pm 0,000004 $ und beim zweiten Durchgang mit der 
Haube $n_\text{Luft,ohne} =  1,000310 \pm 0,000006 $.
Der Theoriewert liegt für Luft bei $n_\text{Luft, Theo} = 1,000292$ \cite{Brechungsindex}.
Die Abweichung von diesem Theoriewert liegt damit bei 13,36\%  bzw. 6,16\% bei Betrachtung der signifikanten Stellen. 
Über alle Messreihen gemittelt, ergibt sich für den Brechungsindex von Luft bei Normatmosphäre $n_\text{Luft} = 1,000320 \pm 0,000004 $, welcher eine 
Abweichung von 9,59\% zum Theoriewert hat. Es ist somit zu sehen, dass die Haube einen Unterschied bewirkt und die Messung ersichtlich genauer werden lässt. 
Da der Brechungsindex von Luft im Allgemeinen sehr nah an 1 liegt, können Störungen wie Luftdruckschwankungen einen relativ starken Einfluss haben.

\begin{table}[h]
    \centering
    \caption{Die ermittelten Brechungsindizes von Glas und Luft im Vergleich zum jeweiligen Theoriewert.}
    \label{tab:ergebnisübersicht}
    \begin{tabular}{c c c c c}
      \toprule
       & $n_\text{Glas}$ & $n_\text{Luft, ohne Haube}$ & $n_\text{Luft, mit Haube}$ & $n_{\text{Luft}}$\\
      \midrule
      Theorie    &  1,45                & 1,000292     &   1,000292    & 1,000292\\   
      Versuch    &  1,55044793          & $1,000331 \pm 0,000004$   &  $1,000310 \pm 0,000006$  & $1,000320 \pm 0,000004$ \\   
      Abweichung &  6,93\%                &   13,36\%                   &   6,16\%        & 9,59\%         \\   
      \bottomrule
    \end{tabular}
  \end{table}