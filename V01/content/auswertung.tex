\section{Auswertung}
\label{sec:Auswertung}

\subsection{Justage}
  \subsubsection{Justieren der Verzögerungsleitungen}
    Wie in der Durchführung beschrieben werden die Verzögerungsleitungen vor den Diskriminatoren justiert.
    Als Verzögerungszeitdifferenz wird $\increment t = t_1 - t_2 = \SI{2}{\nano\second}$ genommen, dieser Wert bleibt über den ganzen Versuch bestehen.
    Dadurch beträgt die Ereignisrate $\SI{18.8}{\per\second}$.
    Die aufgenommenen Messwerte dieser Kalibration sind in \autoref{tab:verzogerung} zu finden.
    Für die Höhe des Plateaus ergibt sich mithilfe einer Mittelwertsrechnung
    \begin{equation*}
      I_\text{Plateau} = 351,57 \pm 18,75 \, .
    \end{equation*}
    Mit den Messwerten des Anstiegs und des Abfalls wird jeweils eine Ausgleichsrechnung mit der Form
    \begin{equation*}
      I = m \cdot t + I_0
    \end{equation*}
    durchgeführt.
    Mithilfe von python werden somit folgende Fitparameter bestimmt:
    \begin{align*}
      &\text{Anstieg} &m_\text{Ab}=43,00 \pm 1,72   &&b_\text{Ab}= 330,65 \pm 6,68 \, ,\\
      &\text{Abfall}  &m_\text{An}=-34,87 \pm 1,44  &&b_\text{An}= 399,48 \pm 11,33 \, .\\
    \end{align*}
    Die jeweiligen Ausgleichsgeraden sind ebenfalls in \autoref{fig:Justage} zu finden.
    Die Halbwertsbreite des Plateaus wird über die Schnittstellen der beiden Ausgleichsgeraden mit der halben Plateauhöhe bestimmt.
    Daraus ergibt sich für die beiden Schnittstellen
    \begin{align*}
      t_\text{links} = \SI{0.49(16)}{\nano\second} && t_\text{rechts}= \SI{1.37(33)}{\nano\second} \, . \\
    \end{align*}
    Dadurch folgt für die Halbwertsbreite und somit auch für die doppelte Breite der Rechteckimpulse
    \begin{equation*}
      t_\sfrac{1}{2} = \SI{1.9(4)}{\nano\second} \, .
    \end{equation*}
    \begin{figure}[h]
      \centering
      \includegraphics[width=\textwidth]{build/Verzoegerungsleitung.pdf}
      \caption{Ausgleichsrechnung und Plateau.}
      \label{fig:Justage}
    \end{figure}
    \begin{table}[h]
      \centering
      \caption{Verzögerungsmessung.}
      \label{tab:verzogerung}
      \begin{tabular}{c c}
        \toprule
         & $\increment t \, [\si{\nano\second}]$ & Counts pro 20 Sekunden\\
        \midrule
        -10.0 &      1  \\
        -9.0  &      4  \\
        -8.0  &      11 \\
        -7.0  &      21 \\
        -6.0  &      75 \\
        -5.5  &      103\\
        -5.0  &      109\\
        -4.5  &      121\\
        -4.0  &      163\\
        -3.5  &      205\\
        -3.0  &      196\\
        -2.5  &      230\\
        -2.0  &      228\\
        -1.5  &      289\\
        -1.0  &      288\\
        -0.5  &      304\\
        0     &      319\\
        0.5   &      340\\
        1.0   &      372\\
        1.5   &      356\\
        2.0   &      345\\
        2.5   &      344\\
        3.0   &      352\\
        3.5   &      273\\
        4.0   &      352\\
        4.5   &      230\\
        5.0   &      242\\
        5.5   &      219\\
        6.0   &      199\\
        7.0   &      155\\
        8.0   &      109\\
        9.0   &      72 \\
        10.0  &      40 \\
        11.0  &      13 \\
        12.0  &      0  \\
        \bottomrule
      \end{tabular}
    \end{table}
  
  \subsubsection{Kalibration}
  Um mit dem Mutlichannelanalyzer arbeiten zu können, muss festgestellt werden welcher Kanal für welche Zerfallszeit steht.
  Durch die Kalibrationsmessung kann jedem Kanal eine Zerfallszeit zugeordnet werden.
  Die aufgenommen Messwerte sind in \autoref{tab:Kalibration_MCA} zu finden.
  Durch die Messwerte wird eine Ausgleichsrechnung durchgeführt mit der Form
  \begin{align*}
    ch = m \cdot t + ch_0 \, .
  \end{align*}
  Der letzte Messwert zu $\increment t = \SI{9.9}{\micro\second}$ wird in der Ausgleichsrechnung nicht berücksichtigt, da dieser die Ausgleichsgerade fälschlicherweise verschiebt.
  Da anscheindend dieser Zeitabstand zwischen den beiden Impulsen schon mit der Suchzeit $T_S = \SI{10}{\micro\second}$ gleichgesetzt wird und somit keine Impulse in diesem Channel zu sehen sind.
  Dabei werden folgende Werte ermittelt:
  \begin{align*}
    m = 0,0217 \pm 8,9686 && ch_0 = 0,1507 \pm 0,0023\, . \\ 
  \end{align*}
  Eine grafische Darstellung der Messwerte sowie die dazugehörige Ausgleichsrechnung ist in \autoref{fig:Kalibration_MCA} zu finden.
  \begin{figure}
    \centering
    \includegraphics[width=\textwidth]{build/Kalibration_MCA.pdf}
    \caption{Kalibration MCA}
    \label{fig:Kalibration_MCA}
  \end{figure}
  \begin{table}[h]
    \centering
    \caption{Kalibration MCA.}
    \label{tab:Kalibration_MCA}
    \begin{tabular}{c c c}
      \toprule
       & $\increment t \, [\si{\nano\second}]$& Channel & Counts pro 15 Sekunden\\
      \midrule
      0.3  && 7 	   && 16213 \\
      0.6  && 21 	   && 15462 \\
      0.9  && 35 	   && 15547 \\
      1.2  && 48 	   && 13928 \\
      1.5  && 62 	   && 15391 \\
      1.8	 && 76 	   && 15293 \\ 
      2.1	 && 90 	   && 15470 \\
      2.4  && 103.98 && 15172 \\
      2.7  && 117 	 && 15234 \\
      3.0  && 131 	 && 16310 \\
      3.3  && 145 	 && 15505 \\
      3.6  &&	159 	 && 15470 \\
      3.9  && 173 	 && 15405 \\
      4.2  && 186 	 && 15301 \\
      4.5  &&	200 	 && 15206 \\
      4.8  && 214 	 && 15221 \\
      5.1  && 228 	 && 15780 \\
      5.4  && 242 	 && 15224 \\
      5.7  && 256 	 && 15158 \\
      6.0  && 269 	 && 15125 \\
      6.3  && 283 	 && 15334 \\
      6.6  && 297 	 && 16094 \\
      6.9  && 311 	 && 15216 \\
      7.2  && 325 	 && 15094 \\
      7.5  && 338.62 && 15517 \\
      7.8  && 352 	 && 15390 \\
      8.1  && 366 	 && 15162 \\
      8.4  && 380 	 && 15348 \\
      8.7  && 394 	 && 16517 \\
      9.0  && 408 	 && 15362 \\
      9.3  && 421.99 && 15254 \\
      9.6  && 435 	 && 15454 \\
      9.9  && 0 	   &&  0    \\
      \bottomrule
    \end{tabular}
  \end{table}

\subsection{Messung der Lebensdauer}
  \subsubsection{Theoretische Untergrundrate}
    Neben Myonenzerfälle, die Stoppsignale auslösen, können weitere Myoneneinfälle dieses Stoppsignal auslösen.
    Dieser Untergrund trifft auf alle Channels gleichmäßig auf und läuft über die ganze Messung ab.
    Dafür wird \eqref{eqn:poisson} verwendet.
    Der Erwartungswert $\mu$ sowie die durschnitltlich gemessene Rate $I_\text{Mess}$ wird mit \eqref{eqn:mu} bestimmt.
    Die Suchzeit $T_\text{S}$ wurde auf $\SI{10}{\micro\second}$ eingestellt.
    Dabei beträgt die Messzeit
    \begin{equation*}
      t_\text{Mess} = \SI{272201}{\second}
    \end{equation*}
    und die gesamte Anzahl der Startsignale
    \begin{equation*}
      N_\text{Start} = 4560816 \pm 2135,61 \, .
    \end{equation*}
    Somit ergibt sich für die durchschnittlich gemessene Rate
    \begin{equation*}
      I_\text{Mess} = 16,755 \pm 0,008 \, .
    \end{equation*}
    Dadurch kann mithilfe von \eqref{eqn:Untergrund} der gesamte Untergrund während der Messzeit bestimmt werden.
    Dieser beträgt somit
    \begin{equation*}
      U_\text{ges} = 764,1 \pm 0,7 \,.
    \end{equation*}
    Der gesamte Untergrund wird auf die relevanten Channels normiert.
    Aus der Kalibrationsmessung ergeben sich als relevante Channels innerhalb der Suchzeit $T_\text{S}$ das Intervall $[4;445]$.
    Somit beträgt der normierte Untergrund
    \begin{equation*}
      U_\text{norm} = 1,7325 \pm 0,0016 \, .
    \end{equation*}
  
  \subsubsection{Bestimmung der Lebensdauer kosmischer Myonen - numerisch}
    Die aufgenommenen Messwerte sind in \autoref{fig:eFkt_numerisch} zu sehen.
    Durch diese Messwerte wird ein Fit der Form
    \begin{equation*}
      N(t) = N_0 \cdot \exp(-\lambda t) +U_\text{n} \, .
    \end{equation*}
    Dabei werden die ersten vier Channels sowie die Channels > 445 nicht berücksichtigt, da diese Null Counts haben.
    Bei den Channels > 445 liegt es an der Suchzeit $T_\text{S}$, die überschritten wird und somit Signale, die nach der Suchzeit eintreffen nicht als Stoppsignale registriert werden.
    Für die Fitparameter ergeben sich somit folgende Werte:
    \begin{align*}
      N_0 &= 153,38 \pm 1,56 \, ,\\
      \lambda &= \SI{0,51(01)}{\per\micro\second} \, ,\\
      U_\text{n} &= 2,64 \pm 0,56 \, . \\
    \end{align*}
    Bei dem hier betrachteten Untergrund $U_\text{n}$ handelt es sich um ein numerisch bestimmten Untergrund.
    Die Fit-Funktion ist in \autoref{fig:eFkt_numerisch} zu finden.
    Durch die bestimmte Zerfallskonstante $\lambda$ kann nun die Lebensdauer berechnet werden.
    Es gilt
    \begin{equation*}
      \tau_\text{num} = \frac{1}{\lambda} = \SI{1.97(04)}{\micro\second} \, .
    \end{equation*}
    \begin{figure}
      \centering
      \includegraphics[width=\textwidth]{build/Messung_Zeit_Counts_num.pdf}
      \caption{Messwerte und eFunktion-Fit.}
      \label{fig:eFkt_numerisch}
    \end{figure}

  \subsubsection{Bestimmung der Lebensdauer kosmischer Myonen - theoretisch}
    Zunächst wird von den signifikanten Messwerten die theoretisch bestimmte Untergrundrate abgezogen.
    Diese beträgt
    \begin{equation*}
      U_\text{norm} = 1,7325 \pm 0,0016 \, .
    \end{equation*}
    Durch die angepassten Messwerte wird eine Ausgleichsrechnung der Form
    \begin{equation*}
      N(t) = N_0 \cdot \exp(-\lambda t)
    \end{equation*}
    durchgeführt.
    Für die Fitparameter ergebn sich somit folgende Werte:
    \begin{align*}
      N_0 &= 153,06 \pm 1,53 \, \text{und}\\
      \lambda &= \SI{0.496(006)}{\per\micro\second} \, . \\
    \end{align*}
    Aus der Zerfallskonstante ergibt sich die Lebensdauer wie folgt:
    \begin{equation*}
      \tau_\text{theo} = \frac{1}{\lambda} = \SI{2.017(026)}{\micro\second} \, .
    \end{equation*}
    Die bereinigten Messwerte sowie die Fit-Funktion sind in \autoref{fig:eFkt_theo} zu finden.
    \begin{figure}
      \centering
      \includegraphics[width=\textwidth]{build/Messung_Zeit_Counts_theo.pdf}
      \caption{Messwerte und eFunktion-Fit.}
      \label{fig:eFkt_theo}
    \end{figure}