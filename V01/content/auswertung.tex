\section{Auswertung}
\label{sec:Auswertung}

\subsection{Justage}
  \subsubsection{Justieren der Verzögerungsleitungen}
    Wie in der Durchführung beschrieben werden die Verzögerungsleitungen vor den Diskriminatoren justiert. 
    Die aufgenommenen Messwerte dieser Kalibration sind in \autoref{tab:verzogerung} zu finden.
    Für die Höhe des Plateaus ergibt sich mithilfe einer Mittelwertsrechnung
    \begin{equation*}
      I_\text{Plateau} = \num{351.57 \pm 18.75} \, .
    \end{equation*}
    Mit den Messwerten des Anstiegs und des Abfalls wird jeweils eine Ausgleichsrechnung mit der Form
    \begin{equation*}
      I = m \cdot t + b
    \end{equation*}
    durchgeführt.
    Mithilfe von python werden somit folgende Fitparameter bestimmt:
    \begin{align*}
      &\text{Anstieg} &m_\text{An}=\SI{43.00 \pm 1.72}{\per\nano\second}   &&b_\text{An}= \num{330.65 \pm 6.68} \, ,\\
      &\text{Abfall}  &m_\text{Ab}= \SI{-34.87 \pm 1.44}{\per\nano\second}  &&b_\text{Ab}= \num{399.48 \pm 11.33} \, .
    \end{align*}
    Die jeweiligen Ausgleichsgeraden sind ebenfalls in \autoref{fig:Justage} zu finden.
    Die Halbwertsbreite des Plateaus wird über die Schnittstellen der beiden Ausgleichsgeraden mit der halben Plateauhöhe bestimmt.
    Daraus ergibt sich für die beiden Schnittstellen
    \begin{align*}
      t_\text{links} = \SI{-3.60(21)}{\nano\second} && t_\text{rechts}= \SI{6.4(4)}{\nano\second} \, .
    \end{align*}
    Dadurch folgt für die Halbwertsbreite und somit auch für die doppelte Breite der Rechteckimpulse
    \begin{equation*}
      t_\frac{1}{2} = \SI{2.8(5)}{\nano\second} \, .
    \end{equation*}
    Als Verzögerungszeitdifferenz wird $\increment t = \SI{2}{\nano\second}$ genommen, dieser Wert bleibt über den ganzen Versuch bestehen.
    Dadurch beträgt die Ereignisrate $\SI{18.8}{\second\tothe{-1}}$.
    \begin{figure}[h]
      \centering
      \includegraphics[width=0.8\textwidth]{build/Verzoegerungsleitung.pdf}
      \caption{Eine graphische Darstellung der Messdaten der Verzögerungszeitmessung. Es werden ein Anstieg, ein Plateau und ein Abfall identifiziert und 
      entsprechend Ausgleichsrechnungen gemacht.}
      \label{fig:Justage}
    \end{figure}
    \begin{table}[h]
      \centering
      \caption{Die Messdaten der Verzögerungsmessung. Es wurde für jede Verzögerungszeit $\SI{20}{\second}$ lang gemessen.}
      \label{tab:verzogerung}
      \begin{tabular}{S[table-format=2.1] c}
        \toprule
         {$\increment t \,\mathbin{/} \si{\nano\second}$} & {Impulse $I$} \\
        \midrule
        -10.0 &      1  \\
        -9.0  &      4  \\
        -8.0  &      11 \\
        -7.0  &      21 \\
        -6.0  &      75 \\
        -5.5  &      103\\
        -5.0  &      109\\
        -4.5  &      121\\
        -4.0  &      163\\
        -3.5  &      205\\
        -3.0  &      196\\
        -2.5  &      230\\
        -2.0  &      228\\
        -1.5  &      289\\
        -1.0  &      288\\
        -0.5  &      304\\
        0     &      319\\
        0.5   &      340\\
        1.0   &      372\\
        1.5   &      356\\
        2.0   &      345\\
        2.5   &      344\\
        3.0   &      352\\
        3.5   &      273\\
        4.0   &      352\\
        4.5   &      230\\
        5.0   &      242\\
        5.5   &      219\\
        6.0   &      199\\
        7.0   &      155\\
        8.0   &      109\\
        9.0   &      72 \\
        10.0  &      40 \\
        11.0  &      13 \\
        12.0  &      0  \\
        \bottomrule
      \end{tabular}
    \end{table}
  
  \subsubsection{Kalibration}
  Um mit dem Mutlichannelanalyzer arbeiten zu können, muss festgestellt werden welcher Kanal für welche Zerfallszeit steht.
  Durch die Kalibrationsmessung kann jedem Kanal eine Zerfallszeit zugeordnet werden. Es werden für die verschiedenen Zeitabstände $\increment t$
  innerhalb von $\SI{15}{\second}$ gemessen, wie viele Ereignisse in einen Kanal fallen, und diese beiden Werte notiert. Falls es Ereignisse in mehreren 
  Kanälen gibt, dann wird ein gewichteter Mittelwert ausgerechnet. 
  Die aufgenommen Messwerte sind in \autoref{tab:Kalibration_MCA} zu finden.
  Durch die Messwerte wird eine Ausgleichsrechnung durchgeführt mit der Form
  \begin{align*}
    \text{ch} = m \cdot t + \text{ch}_0 \, .
  \end{align*}
  Der letzte Messwert zu $\increment t = \SI{9.9}{\micro\second}$ wird in der Ausgleichsrechnung nicht berücksichtigt, da dieser die Ausgleichsgerade fälschlicherweise verschiebt.
  Da anscheinend dieser Zeitabstand zwischen den beiden Impulsen schon mit der Suchzeit $T_\text{S} = \SI{10}{\micro\second}$ gleichgesetzt wird und somit keine Impulse in diesem Kanal zu sehen sind.
  Dabei werden folgende Werte ermittelt:
  \begin{align*}
    m = \SI{0.0217 \pm 8.9686}{\per\micro\second} && \text{ch}_0 = \num{0.1507 \pm 0.0023}\, .
  \end{align*}
  Eine grafische Darstellung der Messwerte sowie die dazugehörige Ausgleichsrechnung ist in \autoref{fig:Kalibration_MCA} zu finden.
  \begin{figure}
    \centering
    \includegraphics[width=0.8\textwidth]{build/Kalibration_MCA.pdf}
    \caption{Die Messwerte der Messung zur Kalibration des MCA mit einer Ausgleichsrechnung.}
    \label{fig:Kalibration_MCA}
  \end{figure}
  \begin{table}[h]
    \centering
    \caption{Die Messwerte zur Kalibration des MCAs. Jeder Messpunkt wurde für $\SI{15}{\second}$ gemessen.}
    \label{tab:Kalibration_MCA}
    \begin{tabular}{S[table-format=1.1] S[table-format=3.2] c}
      \toprule
      {$\increment t \, \mathbin{/} \si{\nano\second}$}& Kanal & {Impulse $I$}\\
      \midrule
      0.3  & 7 	   & 16213 \\
      0.6  & 21 	 & 15462 \\
      0.9  & 35 	 & 15547 \\
      1.2  & 48 	 & 13928 \\
      1.5  & 62 	 & 15391 \\
      1.8	 & 76 	 & 15293 \\ 
      2.1	 & 90 	 & 15470 \\
      2.4  & 103.98 & 15172 \\
      2.7  & 117 	 & 15234 \\
      3.0  & 131 	 & 16310 \\
      3.3  & 145 	 & 15505 \\
      3.6  &	159  & 15470 \\
      3.9  & 173 	 & 15405 \\
      4.2  & 186 	 & 15301 \\
      4.5  &	200  & 15206 \\
      4.8  & 214 	 & 15221 \\
      5.1  & 228 	 & 15780 \\
      5.4  & 242 	 & 15224 \\
      5.7  & 256 	 & 15158 \\
      6.0  & 269 	 & 15125 \\
      6.3  & 283 	 & 15334 \\
      6.6  & 297 	 & 16094 \\
      6.9  & 311 	 & 15216 \\
      7.2  & 325 	 & 15094 \\
      7.5  & 338.62 & 15517 \\
      7.8  & 352 	 & 15390 \\
      8.1  & 366 	 & 15162 \\
      8.4  & 380 	 & 15348 \\
      8.7  & 394 	 & 16517 \\
      9.0  & 408 	 & 15362 \\
      9.3  & 421.99 & 15254 \\
      9.6  & 435 	 & 15454 \\
      9.9  & 0 	   &  0    \\
      \bottomrule
    \end{tabular}
  \end{table}

\subsection{Messung der Lebensdauer}
  \subsubsection{Theoretische Untergrundrate}
    Neben Myonenzerfälle, die Stoppsignale auslösen, können weitere Myoneneinfälle dieses Stoppsignal auslösen.
    Dieser Untergrund trifft auf alle Kanäle gleichmäßig auf und läuft über die ganze Messung ab.
    Dafür wird Gleichung \eqref{eqn:poisson} verwendet.
    Der Erwartungswert $\mu$ sowie die durschnittlich gemessene Rate $I_\text{Mess}$ wird mit \eqref{eqn:mu} bestimmt.
    Die Suchzeit $T_\text{S}$ wurde auf $\SI{10}{\micro\second}$ eingestellt.
    Dabei beträgt die Messzeit
    \begin{equation*}
      t_\text{Mess} = \SI{272201}{\second}
    \end{equation*}
    und die gesamte Anzahl der Startsignale
    \begin{equation*}
      N_\text{Start} = \num{4560816 \pm 2135.61} \, .
    \end{equation*}
    Somit ergibt sich für die durchschnittlich gemessene Rate
    \begin{equation*}
      I_\text{Mess} = \SI{16.755 \pm 0.008}{\per\second} \, .
    \end{equation*}
    Dadurch kann mithilfe von \eqref{eqn:Untergrund} der gesamte Untergrund während der Messzeit bestimmt werden.
    Dieser beträgt somit
    \begin{equation*}
      U_\text{ges} = \num{764.1 \pm 0.7} \,.
    \end{equation*}
    Der gesamte Untergrund wird auf die relevanten Kanäle normiert.
    Aus der Kalibrationsmessung ergeben sich als relevante Kanäle innerhalb der Suchzeit $T_\text{S}$ das Intervall $[4;445]$.
    Somit beträgt der normierte Untergrund
    \begin{equation*}
      U_\text{norm} = \num{1.7325 \pm 0.0016} \, .
    \end{equation*}
  
  \subsubsection{Bestimmung der Lebensdauer - numerisch}
    Die aufgenommenen Messwerte sind in \autoref{fig:eFkt_numerisch} zu sehen.
    Durch diese Messwerte wird ein Fit der Form
    \begin{equation*}
      N(t) = N_0 \cdot \symup{e}^{-\lambda t} +U_\text{num} \, .
    \end{equation*}
    Dabei werden die ersten vier Kanäle sowie die Kanäle > 445 nicht berücksichtigt, da diese Null Counts haben.
    Bei den Kanäle > 445 liegt es an der Suchzeit $T_\text{S}$, die überschritten wird und somit Signale, die nach der Suchzeit eintreffen nicht als Stoppsignale registriert werden.
    Für die Fitparameter ergeben sich somit folgende Werte:
    \begin{align*}
      N_0 &= \num{153.38 \pm 1.56} \, ,\\
      \lambda &= \SI{0.51(01)}{\per\micro\second} \, ,\\
      U_\text{num} &= \num{2.64 \pm 0.56} \, .
    \end{align*}
    Bei dem hier betrachteten Untergrund $U_\text{n}$ handelt es sich um ein numerisch bestimmten Untergrund.
    Die Fit-Funktion ist in \autoref{fig:eFkt_numerisch} zu finden.
    Durch die bestimmte Zerfallskonstante $\lambda$ kann nun die Lebensdauer berechnet werden.
    Es gilt
    \begin{equation*}
      \tau_\text{num} = \frac{1}{\lambda} = \SI{1.97(04)}{\micro\second} \, .
    \end{equation*}
    \begin{figure}[h]
      \centering
      \includegraphics[width=0.8\textwidth]{build/Messung_Zeit_Counts_num.pdf}
      \caption{Die aufgenommenen Messwerte zur Besimmung der Lebensdauer, sowie der dazugehörige Fit mit einer e-Funktion.
                Die schwarz makierten Messpunkte wurden bei der Ausgleichsrechnung nicht berücksichtigt, da diese teilweise aufgrund der überschrittenen Suchzeit keine Ereignisse haben.}
      \label{fig:eFkt_numerisch}
    \end{figure}

  \subsubsection{Bestimmung der Lebensdauer - theoretisch}
    Zunächst wird von den signifikanten Messwerten die theoretisch bestimmte Untergrundrate abgezogen.
    Diese beträgt
    \begin{equation*}
      U_\text{theo} = \num{1.7325 \pm 0.0016} \, .
    \end{equation*}
    Durch die angepassten Messwerte wird eine Ausgleichsrechnung der Form
    \begin{equation*}
      N(t) = N_0 \cdot \exp(-\lambda t)
    \end{equation*}
    durchgeführt.
    Für die Fitparameter ergebn sich somit folgende Werte:
    \begin{align*}
      N_0 &= \num{153.06 \pm 1.53} \, \text{und}\\
      \lambda &= \SI{0.496(006)}{\per\micro\second} \, .
    \end{align*}
    Aus der Zerfallskonstante ergibt sich die Lebensdauer wie folgt:
    \begin{equation*}
      \tau_\text{theo} = \frac{1}{\lambda} = \SI{2.017(026)}{\micro\second} \, .
    \end{equation*}
    Die bereinigten Messwerte sowie die Fit-Funktion sind in \autoref{fig:eFkt_theo} zu finden.
    \begin{figure}[h]
      \centering
      \includegraphics[width=0.8\textwidth]{build/Messung_Zeit_Counts_theo.pdf}
      \caption{Die aufgenommenen Messwerte zur Bestimmung der Lebensdauer, sowie der dazugehörige Fit mit einer e-Funktion.
                Die schwarz makierten Messpunkte wurden bei der Ausgleichsrechnung nicht berücksichtigt, da diese teilweise aufgrund der überschrittenen Suchzeit keine Ereignisse haben.}
      \label{fig:eFkt_theo}
    \end{figure}