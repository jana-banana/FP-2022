\section{Diskussion}
\label{sec:Diskussion}
Die Versuchsdurchführung verlief ohne Probleme.
Bei der Justage der Verzögerungsleitung wurde eine Zeitdifferenz von
\begin{align*}
    T_\text{V} = \SI{2}{\nano\second}
\end{align*}
gewählt.
Diese liegt relativ mittig auf dem Plateau.
Auch bei der Kalibration des MCAs gab es keine weiteren Probleme.
Neben dem erwarteten linearen Verlauf zwischen Zeit und Channel, wurde auch festgestellt, dass die gemessene Anzahl an Ereignissen in derselben Größenordnung waren.
Desweiteren wurde die Empfindlichkeit des Monoflop bemerkt.
Bei einem zeitlichen Abstand von $\SI{9.9}{\micro\second}$ wurden keine Ereignisse registriert, da dieser Abstand mit der Suchzeit $T_\text{S} = \SI{10}{\micro\second}$ gleichgesetzt wurde.

\noindent
Die beiden bestimmten Untergrundraten betragen
\begin{align*}
    U_\text{theo} = 1,7325 \pm 0,0016 &&       U_\text{num} &= 2,64 \pm 0,56 \, . \\
\end{align*}
Die relativ Abweichung der Untergrundraten voneinander beträgt somit
\begin{align*}
    \increment U = 1,52 \, also \, 56\% ? \, \\
\end{align*} %U_num/U_theo
somit liegen diese nah beieinander.
Wird die Bestimmung der Untergrundraten betrachtet, fällt auf, dass der theoretisch bestimmte Untergrund weniger fehleranfällig ist, da die Werte zur Bestimmung relativ eindeutig ist.
Die numerisch bestimmten Untergrundrate basiert allein auf eine Ausgleichsrechnung, die stark variiert, insbesondere wenn entschieden werden muss, welche Messwerte signifikant sind und welche außer Acht gelassen werden müssen.

\noindent
Die daraus berechnte Lebensdauer beträgt
\begin{align*}
    \tau_\text{theo} = \SI{2.017(26)}{\micro\second} && \tau_\text{num} = \SI{1.97(4)}{\micro\second} \, .
\end{align*}
Die relative Abweichung zueinander beträgt
\begin{align*}
    \increment \tau = 0,976 \, also \, 2,4\% ? \, . \\
\end{align*}%tau_num/tau_theo
Der Literaturwert für die Lebensdauer kosmischer Myonen beträgt $\SI{2.2}{\micro\second}$ \cite{pdg}.
Somit folgen für die Abweichungen der jeweils ermittelten Lebensdauern vom Literautwert:
\begin{align*}
    \increment \tau_\text{theo} = 0.917 -> 8,3\% && \increment \tau_\text{num} = 0.895 -> 10.5\% \, .
\end{align*}
Die Lebensdauer, die mit der theoretischen Untergrundrate bestimmt wurde, ist somit genauer.
Dadurch wird auch klar, dass die theoretische Untergrundrate selbst genauer ist.
Weitere Probleme....