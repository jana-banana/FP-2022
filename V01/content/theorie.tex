\section{Ziel des Versuches}
\label{sec:ziel}

    \noindent In diesem Versuch wird mithilfe eines Szintillationsdetekors die Lebensdauer kosmischer Myonen bestimmt. 
    Dabei sollen die Funktionsweisen der einzelnen Bauteile der Schaltung näher betrachtet werden. 

\section{Theoretische Grundlagen}
\label{sec:Theorie}

    \noindent Damit die Durchführung des Versuches sinnvoll erscheint, werden in den nächsten Abschnitten einige theoretischen Grundlagen erläutert. 
    Dabei geht es zuerst um das Myon, dann um die Lebensdauer von instabilen Teilchen, wo auch das Zerfallsgesetz hergeleitet wird. Abschließend 
    wird die Berechnung der Untergrundrate erläutert. 

    \subsection{Das Myon}

        \noindent Das Myon gehört zu den Elementarteilchen aus dem Standardmodell. Es ist ein geladenes Lepton der 2. Generation und wird umgangsprachlich 
        auch als \enquote{schwerer Bruder} des Elektrons bezeichnet. Es hat also eine elektrische Ladung von -$\symup{e}$ und einen Spin von $\frac{1}{2}$. Da es keine 
        Farbladung trägt, welchselwirkt es über die elektromagnetische und die schwache Wechselwirkung. Die Masse des Myon beträgt etwa $m = \SI{105.6583745(24)}{\mega\electronvolt}$ 
        \cite{pdg}, was ungefähr $207 \cdot m_{e^-}$, also $207$ mal der Masse des Elektrons entspricht.   

        \noindent Kosmische Myonen entstehen in einer Höhe von $\num{10}$\, -\, $\SI{15}{\kilo\metre}$ in der Atmosphäre. Bei dem Zusammentreffen der Atomkerne der Atmosphärenteilchen und der primären 
        kosmischen Strahlung entstehen unter anderem leichte Mesonen, wie Pionen ($\pi$) und Kaonen ($K$). Diese zerfallen aufgrund ihrer leichten Masse nur in Leptonen. Da der gesamte 
        Zerfall von Pionen in Leptonen durch Helizität unterdrückt ist, die Unterdrückung aber antiproportional zur Masse des entstehenden Leptons ist, wird der Zerfall von Pion ins 
        Elektron mit entsprechendem Neutrino deutlich stärker unterdrückt, als es beim Myon mit entsprechendem Neutrino der Fall ist. So entstehen die meisten kosmischen Myonen aus den 
        Zerfällen: 
        \begin{align*}
            \pi^+ &\rightarrow \mu^+ + \nu_{\mu} & \pi^- &\rightarrow \mu^- + \bar{\nu}_{\mu}
        \end{align*}

        \noindent Da Myonen schwerer sind als Elektronen, können und werden sie in diese zerfallen. Dies geschieht mehrheitlich über den Zerfall:
        \begin{align*}
            \mu^+ &\rightarrow \symup{e}^+ + \nu_{\text{e}} + \bar{\nu}_{\mu}   & \mu^- &\rightarrow \symup{e}^- + \bar{\nu}_\text{e} + \nu_{\mu} 
        \end{align*}

        \noindent Myonen haben eine mittlere Lebensdauer von ungefähr $\SI{2.2}{\micro\second}$ \cite{pdg}, sie sind mit nahezu Lichtgeschwindigkeit $c$ unterwegs. Nach klassischer Rechnung sollten 
        sie nicht in dieser Menge auf der Erdoberfläche aufzufinden sein, da sie davor zerfallen. Da die Geschwindigkeit so hoch ist, müssen die Effekte der Zeitdilatation und Längenkontraktion
        beachtet werden. Das Auffinden von Myonen auf der Erdoberfläche wird oft als eines der erste Beispiele genannt bei der Motivation von relativistischen Rechnungen. 

    \subsection{Lebensdauer instabiler Teilchen}

        \noindent Die Lebensdauer eines Teilchens ist über seine Zerfallsbereitschaft definiert. Die Wahrscheinlichkeit, dass ein Teilchen in der Zeit $\text{d}t$ zerfällt ist gegeben 
        durch 
        \begin{equation*}
            \text{d}W = \lambda \cdot \text{d}t \, ,
        \end{equation*}
        wobei $\lambda$ die teilchenspezifische Zerfallskonstante darstellt. Werden nun $N$ Teilchen betrachtet, so ist die Änderung der Teilchenzahl $\text{d}N$ in einem  
        Zeitintervall $\text{d} t$:
        \begin{equation*}
            \text{d} N = -N \cdot \text{d} W = - \lambda  N \cdot \text{d}t \, .
        \end{equation*}
        Diese Gleichung kann integriert werden und daraus ergibt sich das bekannt Zerfallsgesetz 
        \begin{equation*}
            N(t) = N_0 \cdot \symup{e}^{- \lambda \cdot t}\, .
        \end{equation*}
        Als Nenngröße wird im Allgemeinen die mittlere Lebensdauer $\tau = \frac{1}{\lambda}$ angegeben. Dies bezeichnet die Zeit, in welcher der Anteil der Teilchen auf $\frac{1}{\symup{e}}$
        abgesunken ist. 

    \subsection{Das Messverfahren}

        \noindent In diesem Versuch wird die mittlere Lebensdauer kosmischer Myonen aus einer Messung aus Individuallebensdauern ermittelt. In einem Szintillationsdetektor werden Myonen detektiert, 
        falls diese zerfallen, werden auch die Elektronen in dem Detektor registriert. Da sich das Detektorsignal von Elektron und Myon nicht unterscheidet, und Myonen unabhängig voneinander 
        in der gesamten Messzeit in den Tank des Szintillationsdetektor eintreten, kann es passieren, dass statt eines Zerfalls die Zeit zwischen zwei eintretenden Myonen gemessen wird. 
        Daher gibt es gleichmäßig über alle Kanäle verteilt einen Untergrund. \\
        Die Poissonverteilung beschreibt die Wahrscheinlichkeit, dass $n$ Myonen bei einem Erwartungswert von $\mu(T_\text{s})$ mit der Messzeit $T_\text{s}$ eintreffen, gemäß:
        \begin{equation}
            p_{\mu}(n) = \frac{\mu}{n!} \cdot \symup{e}^{- \mu}\, .
            \label{eqn:poisson}
        \end{equation}
        Der Erwartungswert, wie viele Myonen während einer Messzeit $T_\text{s}$ in den Detektor eintreten, ist proportional zu der durchschnittlichen Rate $I_\text{mess}$ mit 
        der Myonen über die gesamte Messzeit in den Tank eingetreten sind. Diese errechnet sich daher aus der gesamten Messzeit $t_\text{mess}$ und der gemessenen Anzahl der 
        eingetretenden Myonen $N_\text{Start}$: 
        \begin{align}
            \mu(T_\text{s}) &= I_\text{mess} \cdot T_\text{s} & \text{mit:} \quad I_\text{mess} &= \frac{N_\text{Start}}{t_\text{mess}} \, .
            \label{eqn:mu}
        \end{align}
        Der Untergrund ist schließlich gegeben durch
        \begin{equation}
            U = N_\text{Start} \cdot p_\text{\mu}(1) \, ,
            \label{eqn:Untergrund}
        \end{equation}
        da genau ein eintretendes Myon in der Messzeit als Elektron aus dem Zerfall gemessen wird und diese Messung verfälscht. Alle weiteren eintretenden Myonen starten eine neue Messung. 