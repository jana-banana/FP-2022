\section{Diskussion}
\label{sec:Diskussion}

Bei der Versuchsdurchführung gab es kleinere Probleme, die die Messwerte beeinflusst haben können.
Zum einen hat sich der Plattenkondensator nicht komplett aufgeladen.
Zwar wurde eine Spannung von $\SI{900}{\volt}$ eingestellt, das Messgerät zeigte aber einen niedrigeren schwankenden Wert an, sodass vermutlich Ladung 
von den Kondensatorplatten abgeflossen ist.
Auch das Ablesen des Depolarisationsstromes gestaltete sich schwierig, da das Picoamperemeter von allen Bewegungen beeinflusst wurde. 
Für kurze Zeit wurde auch nur negativer Strom gemessen.
Bei der ersten Messung wurde die Vakuumpumpe erst später angeschalten, sodass sich mehr Wasserdiple gebildet haben können. Da jedoch der Untergrund 
vor der Berechnung abgezogen wird, sollten die Endergebnisse nicht zu stark beeinflusst sein. 
Beim Kühlen musste etwas Stickstoff nachgekippt werden, da die Temperatur ab $\SI{-40}{\celsius}$ nicht mehr abnahm.

\noindent
Die ermittelten Werte sind in \autoref{tab:ergebnisse} einmal zusammengefasst.
Für die Heizrate wurde für $b_1 = \SI{1.460(4)}{\kelvin\per\minute}$ und für $b_2 = \SI{1.880(5)}{\kelvin\per\minute}$.
Der Theoriewert der Aktivierungsenergie liegt bei $\SI{0.66}{\electronvolt}$, der Wert für die Relaxationszeit bei $\SI{e-18}{\second}$ bis $\SI{e-22}{\second}$ \cite{muccillo}.
Somit liegt eine Abweichung für die Aktivierungsenergie von 1\% (Methode 1) bzw. von ungefähr 25\% bis 30\% (Methode 2) vor.
Bei der Relaxationszeit beträgt die Abweichung ungefähr $10^{10}$ bis $10^{1}$.
Dabei haben die ermittelten Relaxationszeiten bei der zweiten Methode eine sehr viel kleinere Abweichung als bei der ersten Methode.

\noindent
Ein möglicher Grund für die größere Abweichung der Aktivierungsenergie bei der zweiten Methode wäre die Rechnung.
Der Logarithmus wurde angewendet, sowie eine numerische Integration und Kehrwerte wurden gebildet.
Dies führt zu einer ungenaueren Rechnung.
Die Relaxationszeit hat hingegen bei beiden Methoden Abweichungen in derselben Größenordnung.

\begin{table}[h]
    \centering
    \caption{Die ermittelten Werte für die Aktivierungsenergie $W$, sowie die Relaxationszeit $\tau$ sortiert nach Methode 1 und 2.}
    \label{tab:ergebnisse}
    \begin{tabular}{c c c c c}
      \toprule
      &\multicolumn{2}{c}{Methode 1} &\multicolumn{2}{c}{Methode 2}\\
      \cmidrule(lr){2-3}\cmidrule(lr){4-5}
                   &$W \, / \, \si{\electronvolt}$  &$\tau \, / \, \si{\minute}$ & $W \, / \, \si{\electronvolt}$    & $\tau \, / \, \si{\minute}$   \\
      \midrule
      Messung 1    & $0,657 \pm 0,013$              & $(8\pm 5)\cdot 10^{-13}$    & $0,822 \pm 0,012$                 & $(3,7\pm 2,0)\cdot 10^{-16}$  \\   
      Messung 2    & $0,460 \pm 0,040$              & $(0,7\pm 1,2)\cdot 10^{-8}$ & $0,870 \pm 0,011$                 & $(3,1\pm 1,6)\cdot 10^{-17}$  \\   
      \bottomrule
    \end{tabular}
  \end{table}

%tau in seconds:
%tau_11 = 0.100188
%tau_12 = 0.204012
%tau_21 = 0.063972
%tau_22 = 0.055188
%also so ungefähr bei 10^-2