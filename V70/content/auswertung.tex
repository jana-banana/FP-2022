\section{Auswertung}
\label{sec:Auswertung}

\noindent Bei der Versuchsvorbereitung und wird der Rezipient zuerst mit der Drehschieberpumpe evakuiert. Dabei wird der Druck $p_\text{E}$ ermittelt, welcher der kleinst erreichbare Druck 
der Vakuumpumpe ist. Der Wert beträgt $p_\text{E} = \SI{3.9(12)e-3}{\milli\bar}$. Anschließend wird die Turbopumpe eingeschalten und auch von ihr der Enddruck bestimmt, welcher hier 
$p_\text{E} = \SI{1.1(4)e-5}{\milli\bar}$ beträgt. 

\subsection{Turbopumpe}

  \noindent Zuerst wurden die Messungen mit der Turbomolekularpumpe durchgeführt, mit der Leckratenmessung wurde begonnen. Das Volumen des benutzten Rezipienten beträgt $V = \SI{33 \pm 3.3}{L}$. 
  
  \subsubsection{Leckratenmessung}

    \noindent Es wurde, wie in der \autoref{sec:Durchführung} beschrieben ein Gleichgewichtsdruck von $p_\text{G} = \SI{5.0(15)e-5}{\milli\bar}$, $\SI{7.0(21)e-5}{\milli\bar}$, $\SI{1.0(3)e-4}{\milli\bar}$, $\SI{2.0(6)e-4}{\milli\bar}$ eingestellt 
    und bei $t = \SI{0}{\second}$ die Pumpe abgeschiebert und in Schritten von $\increment t = \SI{10}{\second}$ der Druck an dem Messgerät M2 abgelesen. Dies wurde 3 mal für jeden 
    Gleichgewichtsdruck gemacht. Anschließend wurden die Drücke der Messung pro Zeit gemittelt. Die Messwerte der einzelnen Messungen und der gemittelte Druck sind für $p_\text{G} = \SI{1.0(3)e-4}{\milli\bar}$ in der \autoref{tab:turbo_leck_1}, für 
    $\SI{2.0(6)e-4}{\milli\bar}$ in der \autoref{tab:turbo_leck_2}, für $p_\text{G} = \SI{5.0(15)e-5}{\milli\bar}$ in der \autoref{tab:turbo_leck_5} und schließlich für $\SI{7.0(21)e-5}{\milli\bar}$ in der \autoref{tab:turbo_leck_7} zu finden. 
    Die angegebenen Fehler des gemittelten Drucks entstehen aus der Systematik, also der Unsicherheit des Messgerätes. Die statistischen Fehler berechnet sich nach der Gleichung \eqref{eqn:err_arth_middel}, diese werden hier nicht betrachtet, da 
    sie im Vergleich zu den systematischen Fehlern sehr klein sind.

    \begin{figure}[h]
      \begin{subfigure}{0.48\textwidth}
        \centering
        \includegraphics[width=\textwidth]{build/leck_turbo_5e-5mbar.pdf}
        \caption{$p_\text{G}=\SI{5.0(15)e-5}{\milli\bar}$.}
        \label{fig:turbo_leck_50}
      \end{subfigure}
      \hfill
      \begin{subfigure}{0.48\textwidth}
        \centering
        \includegraphics[width=\textwidth]{build/leck_turbo_7e-5mbar.pdf}
        \caption{$p_\text{G}=\SI{7.0(21)e-5}{\milli\bar}$.}
        \label{fig:turbo_leck_70}
      \end{subfigure}
      \hfill
      \begin{subfigure}{0.48\textwidth}
        \centering
        \includegraphics[width=\textwidth]{build/leck_turbo_1e-4mbar.pdf}
        \caption{$p_\text{G}=\SI{1.0(3)e-4}{\milli\bar}$.}
        \label{fig:turbo_leck_100}
      \end{subfigure}
      \hfill
      \begin{subfigure}{0.48\textwidth}
        \centering
        \includegraphics[width=\textwidth]{build/leck_turbo_2e-4mbar.pdf}
        \caption{$p_\text{G}=\SI{2.0(6)e-4}{\milli\bar}$.}
        \label{fig:turbo_leck_200}
      \end{subfigure}
      \caption{Die gemittelten Drücke der Messungen der Drehschieberpumpe zu den verschiedenen Gleichgewichtsdrücken mit der jeweiligen linearen Ausgleichsrechnung.}
      \label{fig:turbo_leck}
    \end{figure}

    \noindent Die gemittelten Drücke werden gegen die Zeit in einem $t-p$ Diagramm dargestellt. Diese sind für die verschiedenen Gleichgewichtsdrücke in der \autoref{fig:turbo_leck} dargestellt. 
    Es werden keine Messdaten exkludiert.  
    Zusätzlich ist jeweils ein Fit der Form $p(t) = m \cdot t + n $ eingezeichnet. Dieser wird mithilfe von python \cite{scipy} gemacht. Die Fitparameter ergeben sich zu den folgenden Werten. 
    \begin{align*}
      \text{Für} \quad  p_\text{G} &= \SI{5.0(15)e-5}{\milli\bar}   & \text{Für} \quad  p_\text{G} &= \SI{7.0(21)e-5}{\milli\bar}\\
      m &= \SI{1.12(5)e-5}{\milli\bar\per\second}                   & m &= \SI{1.93(9)e-5}{\milli\bar\per\second} \\
      n &= \SI{-7.86(3373)e-6}{\milli\bar}                          & n &= \SI{-9.19(611)e-5}{\milli\bar} \\
      \\  
      \text{Für} \quad  p_\text{G} &= \SI{1.0(3)e-4}{\milli\bar} & \text{Für} \quad  p_\text{G} &= \SI{2.0(6)e-4}{\milli\bar}\\
      m &= \SI{5.39(21)e-5}{\milli\bar\per\second}          & m &= \SI{1.92(3)e-4}{\milli\bar\per\second} \\
      n &= \SI{-4.24(146)e-4}{\milli\bar}                & n &= \SI{-3.43(164)e-4}{\milli\bar} 
    \end{align*} 

    \noindent Durch Vergleich mit der Gleichung \eqref{eq:saug_leck_theorie} folgt, dass das Saugvermögen $S$ sich hier durch
    \begin{equation*}
      S = \frac{m \cdot V}{p_\text{G}}
    \end{equation*}
    aus der Steigung der linearen Ausgleichsrechnung berechnet. Es werden folgenden Werte berechnet:
    \begin{align*}
      \text{für} \quad p_\text{G} &= \SI{5.0(15)e-5}{\milli\bar}   & \text{für} \quad p_\text{G} &= \SI{7.0(21)e-5}{\milli\bar} \\
      S &= \SI{7.4 \pm 2.4}{\litre\per\second}                 & S &= \SI{9.1 \pm 3.0}{\litre\per\second}  \\
      \\
      \text{für} \quad p_\text{G} &= \SI{1.0(3)e-4}{\milli\bar}  & \text{für} \quad p_\text{G} &= \SI{2.0(6)e-4}{\milli\bar} \\
      S &= \SI{17.8 \pm 9.2}{\litre\per\second}                & S &= \SI{31.7 \pm 10.1}{\litre\per\second}  
    \end{align*}
    Die Messunsicherheit des Saugvermögens berechnet sich nach der Gleichung \eqref{eqn:err_saug_leck}.

  \subsubsection{Evakuierungskurve}

    \noindent Bei der Messung der Evakuierungskurve der Turbomolekularpumpe wird der Rezipient mit Luft befüllt bis zu einem Druck von $p_0 = \SI{5e-3}{\milli\bar}$. Dann wird der Rezipient 
    abgedichtet und in Abständen von erst $\increment t = \SI{5}{\second}$ und dann $\increment t = \SI{10}{\second}$ wird der Druck abgelesen, bis $ t = \SI{120}{\second}$ erreicht ist. Die Messung wurde drei mal ausgeführt.
    In der weiteren Auswertung wird eine lineare Ausgleichsrechnung gemacht, wobei ein linearer Zusammenhang zwischen dem Logarithmusausdruck 
    \begin{equation*}
      \ln(F) = \ln \left( \frac{p(t) - p_\text{E}}{p_0 - p_\text{E}}\right) = m \cdot t + n \, 
    \end{equation*} 
    und der Zeit besteht, wie es durch Logarithmieren aus der Gleichung \eqref{eqn:Druck_Funktion} hergeleitet wird. Die Messdaten der einzelnen Messungen sind neben dem gemittelten Druck und dem daraus gerechneten Logarithmusausdruck in der \autoref{tab:turbo_eva} aufgelistet. 
    Der Enddruck der Pumpe wurde zu $p_\text{E} = \SI{1.1(4)e-5}{\milli\bar}$ bestimmt. Der Fehler des gemittelten Drucks ergibt sich aus der Systematik, die statistischen Fehler werden nicht betrachtet, da sie deutlich kleiner sind.
    Der Fehler des Logarithmusausdruck berechnet sich nach \eqref{eqn:err_ln}. 
  

    \noindent Es wird der Logarithmusausdruck $\ln(F)$ gegen die Zeit aufgetragen. In den Daten werden lineare Abhängigkeiten gesucht und dann in diesen Bereichen
    eine lineare Ausgleichsrechung der Form 
    \begin{equation*}
      y = m \cdot t + n
    \end{equation*}
    mit Hilfe von \cite{scipy} durchgeführt. Es werden drei Bereiche gesucht. \\
    Dieses Vorgehen ist in der \autoref{fig:evaku_turbo} zu sehen.
    Es wird kein Datenpunkt exkludiert. 
    Die Parameterdaten für die lineare Ausgleichsrechung sind in der \autoref{tab:fitpara_turbo_alt} aufgelistet. \\ 
    

    \begin{figure}[h]
      \centering
      \includegraphics[width=\textwidth]{build/evakuturbo.pdf}
      \caption{Die Messdaten der Evakuierungsmessung der Turbopumpe aufgetragen in einem Logarithmusausdruck gegen die Zeit. Zusätzlich werden lineare Ausgleichsrechungen in linear wirkende Bereiche gelegt.}
      \label{fig:evaku_turbo}
    \end{figure}

    \begin{table}[h]
      \centering
      \caption{Die Fitparameter und die daraus errechneten Saugvermögen für die einzelnen linearen Bereiche.}
      \label{tab:fitpara_turbo_alt}
      \sisetup{table-format=2.2}
      \begin{tabular}{c S[table-format=2.3] @{${}\pm{}$} S[table-format=2.3] S @{${}\pm{}$} S[table-format=1.2] S[table-format=1.1] @{${}\pm{}$} S[table-format=1.1]}
        \toprule
        {Bereiche} & \multicolumn{2}{c}{$m \mathbin{/} \left(\si{1\per\second}\right)$} & \multicolumn{2}{c}{$n$} & \multicolumn{2}{c}{$S \mathbin{/} \left(\si{\litre\per\second}\right)$}\\
        \midrule
        $\SI{3e-5}{\milli\bar} \leq p \leq \SI{5e-3}{\milli\bar}$   & -0.26  & 0.03  & -0.26 & 0.28 & 8.4 & 1.2 \\
        $\SI{2.9e-5}{\milli\bar} \leq p \leq \SI{1.75e-5}{\milli\bar}$ & -0.041 & 0.006 & -4.70 & 0.19 & 1.3 & 0.3 \\
        $\SI{1.74}{\milli\bar} \leq p \leq \SI{1.35e-5}{\milli\bar}$ & -0.014 & 0.001 & -5.84 & 0.04 & 0.5 & 0.1 \\
        \bottomrule
      \end{tabular}
    \end{table}


    \noindent Aus der Steigung der Ausgleichsgeraden kann das Saugvermögen berechnet werden, es gilt der Zusammenhang
    \begin{equation*}
      S = - m \cdot V\, .
    \end{equation*}
    Dies folgt aus der Gleichung \eqref{eqn:saug_eva_theo}. Die entsprechenden Saugvermögen sind auch in der \autoref{tab:fitpara_turbo_alt} aufgelistet. 
    Die Fehler der Werte werden mit der Gleichung \eqref{eqn:err_saug_eva} berechnet.

\subsection{Drehschieberpumpe}

    \noindent Nun werden die gleichen Auswertungsschritte bei den Messwerten der Drehschieberpumpe angewendet. Das Volumen des Rezipienten beträgt nun $V = \SI{34 \pm 3.4}{L}$, da die 
    Drehschieberpumpe hinter die dann ausgeschaltete Turbomolekularpumpe geschalten, die Schläuche und die Pumpe, welche nun zusätzlich evakuierten werden müssen, haben also ein Volumen von $V_\text{Zusatz} = \SI{1.0 \pm 0.1}{L}$. 

    \subsubsection{Leckratenmessung}

    \noindent Analog zum Vorgehen bei der Turbomolekularpumpe wird ein Gleichgewichtsdruck eingestellt. Hier werden die
    Gleichgewichtsdrücke $p_\text{G} = \SI{5.0(15)e-1}{\milli\bar}$, $\SI{10(3)}{\milli\bar}$, $\SI{50(15)}{\milli\bar}$, $\SI{100(50)}{\milli\bar}$ benutzt. Für jeden Gleichgewichtsdruck werden 
    drei Messungen durchgeführt. Der Messzeitraum beträgt $\SI{200}{\second}$, es werden in Schritten von $\increment t = \SI{10}{\second}$ Daten aufgenommen. Über die drei aufgenommenen
    Messungen des Drucks wird das arithmetische Mittel genommen. Der in den Tabellen angegebene Fehler des gemittelten Drucks ergibt sich aus der Systematik, die statistischen Fehler aus der 
    Mittelung berechnen sich nach der Gleichung \eqref{eqn:err_arth_middel}, da diese Fehler deutlich kleiner als die systematischen Fehler sind, werden sie nicht weiter betrachtet.
    Die großen Fehler der Messwerte lassen sich dadurch erklären, dass bei der Durchführung mit dem falschen Messgerät gemessen wurde, welches eigentlich nicht für diesen Druckbereich ausgelegt ist. 
    Die Daten befinden sich für den Gleichgewichtsdruck $p_\text{G} = \SI{5.0(15)e-1}{\milli\bar}$ in der \autoref{tab:dreh_leck_05}, und für $p_\text{G} = \SI{10(3)}{\milli\bar}$ in der 
    \autoref{tab:dreh_leck_10}. Für den Gleichgewichtsdruck $p_\text{G} = \SI{50(15)}{\milli\bar}$ sind die Messdaten in der \autoref{tab:dreh_leck_50} und für $p_\text{G} = \SI{100(50)}{\milli\bar}$
    in der \autoref{tab:dreh_leck_100}.

    \noindent Die gemittelten Drücke werden gegen die Zeit aufgetragen. Dies ist in den Abbildungen in der \autoref{fig:dreh_leck} zu sehen. Dazu sind in lineare Ausgleichsrechnungen 
    mit python \cite{scipy} durchgeführt worden. Die Fitparameter ergeben sich zu:
    \begin{align*}
      \text{Für} \quad  p_\text{G} &= \SI{5.0(15)e-1}{\milli\bar}  & \text{Für} \quad  p_\text{G} &= \SI{10(3)}{\milli\bar}\\
      m &= \SI{1.90(6)e-3}{\milli\bar\per\second}           & m &= \SI{0.23 \pm 0.01}{\milli\bar\per\second} \\
      n &= \SI{0.49 \pm 0.0007}{\milli\bar}                 & n &= \SI{10.12 \pm 0.21}{\milli\bar} \\
      \\
      \text{Für} \quad  p_\text{G} &= \SI{50(15)}{\milli\bar}   & \text{Für} \quad  p_\text{G} &= \SI{100(50)}{\milli\bar}\\
      m &= \SI{2.19 \pm 0.11}{\milli\bar\per\second}        & m &= \SI{5.74 \pm 0.43}{\milli\bar\per\second} \\
      n &= \SI{46.12 \pm 3.14}{\milli\bar}                  & n &= \SI{95.27 \pm 8.04}{\milli\bar} 
    \end{align*}
    Es werden bei den Gleichgewichtsdrücken $p_\text{G} = \SI{50(15)}{\milli\bar}, \, \SI{100(50)}{\milli\bar}$ einige Datenpunkte exkludiert, da der erweiterte Messbereich des Messgerätes
    dann überschritten wird und keine Linearität mehr vorliegt, wie es auch in den Abbildungen zu sehen ist. \\
    Aus den Steigungen der linearen Ausgleichsrechnung werden die Saugvermögen berechnet durch Vergleich mit Gleichung \eqref{eq:saug_leck_theorie}, wobei sich die Messunsicherheit nach Gleichung \eqref{eqn:err_saug_leck} berechnet:
    \begin{align*}
      \text{für} \quad p_\text{G} &= \SI{5.0(15)e-1}{\milli\bar} & \text{für} \quad p_\text{G} &= \SI{10(3)}{\milli\bar} \\
      S &= \SI{0.12 \pm 0.05}{\litre\per\second}               & S &= \SI{0.79 \pm 0.25}{\litre\per\second}  \\
      \\
      \text{für} \quad p_\text{G} &= \SI{50(15)}{\milli\bar}  & \text{für} \quad p_\text{G} &= \SI{100(50)}{\milli\bar} \\
      S &= \SI{1.49 \pm 0.48}{\litre\per\second}               & S &= \SI{1.95 \pm 1.01}{\litre\per\second}  
    \end{align*}

    \begin{figure}[h]
      \begin{subfigure}{0.48\textwidth}
        \centering
        \includegraphics[width=\textwidth]{build/leck_dreh_05mbar.pdf}
        \caption{$p_\text{G} = \SI{5.0(15)e-1}{\milli\bar}$.}
        \label{fig:dreh_leck_05}
      \end{subfigure}
      \hfill
      \begin{subfigure}{0.48\textwidth}
        \centering
        \includegraphics[width=\textwidth]{build/leck_dreh_10mbar.pdf}
        \caption{$p_\text{G} = \SI{10(3)}{\milli\bar}$.}
        \label{fig:dreh_leck_10}
      \end{subfigure}
      \hfill
      \begin{subfigure}{0.48\textwidth}
        \centering
        \includegraphics[width=\textwidth]{build/leck_dreh_50mbar.pdf}
        \caption{$p_\text{G} = \SI{50(15)}{\milli\bar}$.}
        \label{fig:dreh_leck_50}
      \end{subfigure}
      \hfill
      \begin{subfigure}{0.48\textwidth}
        \centering
        \includegraphics[width=\textwidth]{build/leck_dreh_100mbar.pdf}
        \caption{$p_\text{G} = \SI{100(50)}{\milli\bar}$.}
        \label{fig:dreh_leck_100}
      \end{subfigure}
      \caption{Die gemittelten Drücke der Messungen zu den verschiedenen Gleichgewichtsdrücken mit der jeweiligen linearen Ausgleichsrechnung.}
      \label{fig:dreh_leck}
    \end{figure}

  \subsubsection{Evakuierungskurve}

    \noindent Die Drehschieberpumpe hat einen Enddruck von $p_\text{E} = \SI{3.9(12)e-3}{\milli\bar}$. Der Rezipient wird für die Aufnahme der Evakuierungskurve mit der Drehschieberpumpe 
    bis zu einem Druck von $p_0 = \SI{1000 \pm 500}{\milli\bar}$ belüftet. Dann wird die Luftzufuhr abgeschiebert und der Druck in Abhängigkeit der Zeit in 
    Schritten von $\increment t = \SI{10}{\second}$ aufgenommen bis $t = \SI{600}{\second}$ erreicht. Die aufgenommenen Messdaten sind in den Tabelle \ref{tab:dreh_eva} und \ref{tab:dreh_eva2} zu finden. 
    Es wird über die Drücke gemittelt, wobei der in den Tabellen angegebene Fehler sich aus der Systematik ergibt, da die statistischen Fehler deutlich kleiner sind.
    Bei der dritten Messreihe werden die ersten beiden aufgenommenen Datenpunkte exkludiert, da diese so gut wie gleich dem dritten Punkt sind und sich 
    daher vermuten lässt, dass die Zeitnahme begonnen hat, bevor der Rezipient komplett abgedichtet war. Daher hat diese Messreihe zwei Werte weniger, dann wird nur über die 
    erste und zweite Messung gemittelt.
    Dort sind außerdem die Ergbnisse der Rechnung 
    \begin{equation*}
      \ln(F) = \ln \left( \frac{p(t) - p_\text{E}}{p_0 - p_\text{E}}\right)
    \end{equation*}
    eingetragen, da diese in ein $t - \ln(F)$ Diagramm eingetragen werden. Die Fehler dieses Ausdruckes werden mit der Gleichung \eqref{eqn:err_ln} berechnet.  

    \begin{figure}[H]
      \centering
      \includegraphics[width=\textwidth]{build/evakudreh.pdf}
      \caption{Die gemittelten Messdaten der Evakuierungsmessung der Drehschieberpumpe aufgetragen in einem Logarithmusausdruck gegen die Zeit. Zusätzlich werden lineare Ausgleichsrechungen in linear wirkende Bereiche gelegt.}
      \label{fig:evaku_dreh}
    \end{figure}

    \noindent Die Abbildungen werden auf lineares Verhalten untersucht, wobei drei verschiedene Bereiche mit linearem Verhalten identifiziert werden. 
    Dies ist in der \autoref{fig:evaku_dreh} zu sehen.  \\ 
    Die Parameterdaten der linearen Ausgleichsrechnungen sind in der \autoref{tab:fitpara_dreh_alt} für die verschiedenen Druckbereichen aufgelistet. Das Saugvermögen berechnet sich 
    aus der Steigung $m$ durch 
    \begin{equation*}
      S = m \cdot V\, ,
    \end{equation*}
    welches sich durch Vergleich mit Gleichung \eqref{eqn:saug_eva_theo} ergibt.
    Die Werte für das Saugvermögen sind auch in der \autoref{tab:fitpara_dreh_alt} aufgelistet, der Fehler berechnet sich nach der Formel \eqref{eqn:err_saug_leck}. 

    \begin{table}[H]
      \centering
      \caption{Die Fitparameter und die daraus errechneten Saugvermögen für die einzelnen linearen Bereiche von der Evakuierungskurve der Drehschieberpumpe.}
      \label{tab:fitpara_dreh_alt}
      \sisetup{table-format=1.1}
      \begin{tabular}{c S[table-format=2.3] @{${}\pm{}$} S[table-format=1.3]  S[table-format=2.2] @{${}\pm{}$} S[table-format=1.2] S @{${}\pm{}$} S}
        \toprule
        {Bereiche:} & \multicolumn{2}{c}{$m \mathbin{/} \left(\si{1\per\second}\right)$} & \multicolumn{2}{c}{ $n$} & \multicolumn{2}{c}{$S \mathbin{/} \left(\si{\litre\per\second}\right)$}\\
        \midrule
        $\SI{10}{\milli\bar} \leq p \leq \SI{1000}{\milli\bar}$ & -0.037 & 0.002 & -0.19 & 0.10 & 1.6 & 0.2 \\
        $\SI{1.5}{\milli\bar} \leq p \leq \SI{10}{\milli\bar}$  & -0.011 & 0.001 & -3.42 & 0.10 & 0.4 & 0.1 \\
        $\SI{0.4}{\milli\bar} \leq p \leq \SI{1.5}{\milli\bar}$ & -0.004 & 0.001 & -5.27 & 0.03 & 0.1 & 0.1 \\
        \bottomrule
      \end{tabular}
    \end{table}