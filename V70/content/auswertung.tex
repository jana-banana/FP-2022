\section{Auswertung}
\label{sec:Auswertung}

\noindent Bei der Versuchsvorbereitung und wird der Tank zuerst mit der Drehschieberpumpe evakuiert. Dabei wird der Druck $p_\text{E}$ ermittelt, welcher der kleinst erreichbare Druck 
der Vakuumpumpe ist. Der Wert beträgt $p_\text{E} = \SI{3.85(115)e-3}{\milli\bar}$. Anschließend wird die Turbopumpe eingeschalten und auch von ihr der Enddruck bestimmt, welcher hier 
$p_\text{E} = \SI{10.9 \pm 3.27}{\milli\bar}$ beträgt. 

\subsection{Turbopumpe}

  \noindent Zuerst wurden die Messungen mit der Turbomolekularpumpe durchgeführt, mit der Leckratenmessung wurde begonnen. Das Volumen des benutzten Tankes beträgt $V = \SI{33 \pm 3.3}{\litre}$. 
  
  \subsubsection{Leckratenmessung}

    \noindent Es wurde, wie in der \autoref{sec:Durchführung} beschrieben mithilfe des Ventils V4 ?? ein Gleichgewichtsdruck von $p_\text{G} = \SI{50}{\nano\bar}, \, \SI{70}{\nano\bar}, \,  \SI{100}{\nano\bar},\, \SI{200}{\nano\bar}$ eingestellt 
    und bei $t = \SI{0}{\second}$ die Pumpe abgeschiebert und in Schritten von $\increment t = \SI{10}{\second}$ der Druck an dem Messgerät diwirf?? abgelesen. Dies wurde 3 mal für jeden 
    Gleichgewichtsdruck gemacht. Anschließend wurden die drei Messungen gemittelt. Die Messwerte der einzelnen Messungen und der gemittelte Druck sind für $p_\text{G} = \SI{100}{\nano\bar}\, \SI{200}{\nano\bar}$
    in der \autoref{tab:turbo_leck_1_2} und für $p_\text{G} = \SI{50}{\nano\bar}\, \SI{70}{\nano\bar}$ in der \autoref{tab:turbo_leck_5_7} zu finden. 

    \begin{table}[h]
      \centering
      \caption{Die Messwerte der einzelnen Messungen und der daraus gemittelte Druckwert für die Leckratenmessung der Turbopumpe mit den Gleichgewichtsdruck $P_\text{G} = \SI{100}{\nano\bar}$.}
      \label{tab:turbo_leck_1}
      \sisetup{table-format=4.0}
      \begin{tabular}{S[table-format=3.0] S @{${}\pm{}$} S S @{${}\pm{}$} S S @{${}\pm{}$} S S[table-format=4.2] @{${}\pm{}$} S[table-format=2.2]}
      \toprule
      {$t \mathbin{/} \si{\second} $} & \multicolumn{2}{c}{$p_\text{M1} \mathbin{/} \si{\nano\bar}$} & \multicolumn{2}{c}{$p_\text{M2} \mathbin{/} \si{\nano\bar}$} & \multicolumn{2}{c}{$p_\text{M3} \mathbin{/} \si{\nano\bar}$} & \multicolumn{2}{c}{$p_\text{Mittel} \mathbin{/} \si{\nano\bar}$} \\
      \midrule
      0   &  100 &   30 &  100 &   30 &  100 &   30 &  100.00 &  0.00 \\
      10  &  360 &  108 &  358 &  107 &  388 &  116 &  368.67 &  7.91 \\
      20  &  592 &  177 &  655 &  196 &  618 &  185 &  621.67 & 14.92 \\
      30  & 1010 &  303 & 1020 &  306 & 1010 &  303 & 1013.33 &  2.72 \\
      40  & 1420 &  426 & 1430 &  429 & 1480 &  444 & 1443.33 & 15.15 \\
      50  & 1880 &  564 & 1890 &  567 & 1900 &  570 & 1890.00 &  4.71 \\
      60  & 2530 &  759 & 2500 &  750 & 2560 &  768 & 2530.00 & 14.14 \\
      70  & 3190 &  957 & 3170 &  951 & 3170 &  951 & 3176.67 &  5.44 \\
      80  & 3800 & 1140 & 3740 & 1122 & 3760 & 1128 & 3766.67 & 14.40 \\
      90  & 4620 & 1386 & 4490 & 1347 & 4620 & 1386 & 4576.67 & 35.38 \\
      100 & 5180 & 1554 & 5140 & 1542 & 5160 & 1548 & 5160.00 &  9.43 \\
      110 & 5710 & 1713 & 5600 & 1680 & 5580 & 1674 & 5630.00 & 33.00 \\
      120 & 6400 & 1920 & 6240 & 1872 & 6200 & 1860 & 6280.00 & 49.89 \\
      \bottomrule
      \end{tabular}
    \end{table}

    \begin{table}[h]
      \centering
      \caption{Die Messwerte der einzelnen Messungen und der daraus gemittelte Druckwert für die Leckratenmessung der Turbopumpe mit den Gleichgewichtsdruck
                $p_\text{G} = \SI{200}{\nano\bar}$.}
      \label{tab:turbo_leck_2}
      \sisetup{table-format=5.0}
      \begin{tabular}{S[table-format=3.0] S @{${}\pm{}$} S S @{${}\pm{}$} S S @{${}\pm{}$} S S[table-format=5.2] @{${}\pm{}$} S[table-format=4.2]}
      \toprule
      {$t \mathbin{/} \si{\second} $} & \multicolumn{2}{c}{$p_\text{M1} \mathbin{/} \si{\nano\bar}$} & \multicolumn{2}{c}{$p_\text{M2} \mathbin{/} \si{\nano\bar}$} & \multicolumn{2}{c}{$p_\text{M3} \mathbin{/} \si{\nano\bar}$} & \multicolumn{2}{c}{$p_\text{Mittel} \mathbin{/} \si{\nano\bar}$} \\
      \midrule
      0   &   200 &   60 &   200 &   60 &   200 &   60 &   200.00 &   0.00 \\
      10  &  1130 &  339 &  1270 &  381 &  1480 &  444 &  1293.33 &  83.04 \\
      20  &  2780 &  834 &  3180 &  954 &  3180 &  954 &  3046.67 & 108.87 \\
      30  &  4830 & 1449 &  5430 & 1629 &  5180 & 1554 &  5146.67 & 142.07 \\
      40  &  6440 & 1932 &  7980 & 2394 &  7500 & 2250 &  7306.67 & 371.46 \\
      50  &  8920 & 2676 & 10400 & 3120 & 10100 & 3030 &  9806.67 & 368.82 \\
      60  & 10700 & 3210 & 11900 & 3570 & 11400 & 3420 & 11333.33 & 284.15 \\
      70  & 12200 & 3660 & 13700 & 4110 & 13300 & 3990 & 13066.67 & 366.16 \\
      80  & 13700 & 4110 & 15600 & 4680 & 15200 & 4560 & 14833.33 & 472.19 \\
      90  & 15500 & 4650 & 18000 & 5400 & 17000 & 5100 & 16833.33 & 593.17 \\
      100 & 17500 & 5250 & 20000 & 6000 & 18900 & 5670 & 18800.00 & 590.67 \\
      110 & 19700 & 5910 & 21800 & 6540 & 21400 & 6420 & 20966.67 & 525.64 \\
      120 & 21500 & 6450 & 23600 & 7080 & 22800 & 6840 & 22633.33 & 499.63 \\
      \bottomrule
      \end{tabular}
    \end{table}

    \begin{table}[h]
      \centering
      \caption{Die Messwerte der einzelnen Messungen und der daraus gemittelte Druckwert für die Leckratenmessung der Turbopumpe mit den Gleichgewichtsdruck $P_\text{G} = \SI{50}{\nano\bar}$.}
      \label{tab:turbo_leck_5}
      \sisetup{table-format=4.1}
      \begin{tabular}{S[table-format=3.0] S @{${}\pm{}$} S S @{${}\pm{}$} S S @{${}\pm{}$} S S[table-format=4.2] @{${}\pm{}$} S[table-format=2.2]}
      \toprule
      {$t \mathbin{/} \si{\second} $} & \multicolumn{2}{c}{$p_\text{M1} \mathbin{/} \si{\nano\bar}$} & \multicolumn{2}{c}{$p_\text{M2} \mathbin{/} \si{\nano\bar}$} & \multicolumn{2}{c}{$p_\text{M3} \mathbin{/} \si{\nano\bar}$} & \multicolumn{2}{c}{$p_\text{Mittel} \mathbin{/} \si{\nano\bar}$} \\
      \midrule
        0.0 &    50.0 &  15.0 &   50.0 &  15.0 &    50.0 &  15.0 &   50.00 & 0.00 \\
       10.0 &   178.0 &  53.4 &  165.0 &  49.5 &   164.0 &  49.2 &  169.00 & 3.68 \\
       20.0 &   250.0 &  75.0 &  246.0 &  73.8 &   252.0 &  75.6 &  249.33 & 1.44 \\
       30.0 &   332.0 &  99.6 &  326.0 &  97.8 &   332.0 &  99.6 &  330.00 & 1.63 \\
       40.0 &   419.0 & 125.7 &  412.0 & 123.6 &   416.0 & 124.8 &  415.67 & 1.66 \\
       50.0 &   497.0 & 149.1 &  497.0 & 149.1 &   496.0 & 148.8 &  496.67 & 0.27 \\
       60.0 &   590.0 & 177.0 &  582.0 & 174.6 &   583.0 & 174.9 &  585.00 & 2.05 \\
       70.0 &   689.0 & 206.7 &  686.0 & 205.8 &   693.0 & 207.9 &  689.33 & 1.66 \\
       80.0 &   835.0 & 250.5 &  823.0 & 246.9 &   829.0 & 248.7 &  829.00 & 2.83 \\
       90.0 &   983.0 & 294.9 &  973.0 & 291.9 &   963.0 & 288.9 &  973.00 & 4.71 \\
      100.0 &  1150.0 & 345.0 & 1140.0 & 342.0 &  1140.0 & 342.0 & 1143.33 & 2.72 \\
      110.0 &  1280.0 & 384.0 & 1270.0 & 381.0 &  1280.0 & 384.0 & 1276.67 & 2.72 \\
      120.0 &  1450.0 & 435.0 & 1440.0 & 432.0 &  1440.0 & 432.0 & 1443.33 & 2.72 \\
      \bottomrule
      \end{tabular}
    \end{table}

    \begin{table}[h]
      \centering
      \caption{Die Messwerte der einzelnen Messungen und der daraus gemittelte Druckwert für die Leckratenmessung der Turbopumpe mit den Gleichgewichtsdruck $p_\text{G} = \SI{70}{\nano\bar}$.}
      \label{tab:turbo_leck_5_7}
      \sisetup{table-format=4.1}
      \begin{tabular}{S[table-format=3.0] S @{${}\pm{}$} S S @{${}\pm{}$} S S @{${}\pm{}$} S S[table-format=4.2] @{${}\pm{}$} S[table-format=2.2]}
      \toprule
      {$t \mathbin{/} \si{\second} $} & \multicolumn{2}{c}{$p_\text{M1} \mathbin{/} \si{\nano\bar}$} & \multicolumn{2}{c}{$p_\text{M2} \mathbin{/} \si{\nano\bar}$} & \multicolumn{2}{c}{$p_\text{M3} \mathbin{/} \si{\nano\bar}$} & \multicolumn{2}{c}{$p_\text{Mittel} \mathbin{/} \si{\nano\bar}$} \\
      \midrule
        0.0 &   70.0 &  21.0 &   70.0 &  21.0 &   70.0 &   21.0 &    70.00 &  0.00 \\
       10.0 &  209.0 &  62.7 &  220.0 &  66.0 &  232.0 &   69.6 &   220.33 &  5.42 \\
       20.0 &  323.0 &  96.9 &  328.0 &  98.4 &  347.0 &  104.1 &   332.66 &  5.97 \\
       30.0 &  442.0 & 132.6 &  453.0 & 135.9 &  468.0 &  140.4 &   454.33 &  6.15 \\
       40.0 &  551.0 & 165.3 &  567.0 & 170.1 &  602.0 &  180.6 &   573.33 & 12.30 \\
       50.0 &  591.0 & 177.3 &  745.0 & 223.5 &  797.0 &  239.1 &   711.00 & 50.50 \\
       60.0 &  881.0 & 264.3 &  934.0 & 280.2 & 1010.0 &  303.0 &   941.66 & 30.57 \\
       70.0 & 1120.0 & 336.0 & 1170.0 & 351.0 & 1260.0 &  378.0 &  1183.33 & 33.44 \\
       80.0 & 1310.0 & 393.0 & 1390.0 & 417.0 & 1460.0 &  438.0 &  1386.66 & 35.38 \\
       90.0 & 1500.0 & 450.0 & 1620.0 & 486.0 & 1700.0 &  510.0 &  1606.66 & 47.45 \\
      100.0 & 1760.0 & 528.0 & 1840.0 & 552.0 & 1950.0 &  585.0 &  1850.00 & 44.97 \\
      110.0 & 1970.0 & 591.0 & 2080.0 & 624.0 & 2240.0 &  672.0 &  2096.66 & 64.00 \\
      120.0 & 2260.0 & 678.0 & 2430.0 & 729.0 & 2560.0 &  768.0 &  2416.66 & 70.92 \\
      \bottomrule
      \end{tabular}
    \end{table}

    \noindent Die gemittelten Drücke werden gegen die Zeit in einem $t-p$ Diagramm dargestellt. Diese sind für die verschiedenen Gleichgewichtsdrücke in der \autoref{fig:turbo_leck} dargestellt. 
    Es werden keine Messdaten exkludiert.  
    Zusätzlich ist jeweils ein Fit der Form $p(t) = m \cdot t + n $ eingezeichnet. Dieser wird mithilfe von \cite{scipy} gemacht. Die Fitparameter ergeben sich zu den folgenden Werten. 
    \begin{align*}
      \text{Für} \,  p_\text{G} &= \SI{50}{\nano\bar} & \text{Für} \,  p_\text{G} &= \SI{70}{\nano\bar}\\
      m &= \SI{11.22 \pm 0.48}{\nano\bar\per\second} & m &= \SI{19.28 \pm 0.86}{\nano\bar\per\second} \\
      n &= \SI{-7.86 \pm 33.73}{\nano\bar} & n &= \SI{-91.84 \pm 61.03}{\nano\bar} \\
      \\
      \text{Für} \,  p_\text{G} &= \SI{100}{\nano\bar} & \text{Für} \,  p_\text{G} &= \SI{200}{\nano\bar}\\
      m &= \SI{53.94(207)}{\nano\bar\per\second} & m &= \SI{191.95 \pm 2.31}{\nano\bar\per\second} \\
      n &= \SI{-424.08 (14611)}{\nano\bar} & n &= \SI{-342.78 \pm 163.26}{\nano\bar} 
    \end{align*} 

    \begin{figure}[h]
      \begin{subfigure}{0.48\textwidth}
        \centering
        \includegraphics[width=\textwidth]{build/leck_turbo_50nbar.pdf}
        \caption{Gleichgewichtsdruck $p_\text{G} = \SI{50}{\nano\bar}$.}
        \label{fig:turbo_leck_50}
      \end{subfigure}
      \hfill
      \begin{subfigure}{0.48\textwidth}
        \centering
        \includegraphics[width=\textwidth]{build/leck_turbo_70nbar.pdf}
        \caption{Gleichgewichtsdruck $p_\text{G} = \SI{70}{\nano\bar}$.}
        \label{fig:turbo_leck_70}
      \end{subfigure}
      \hfill
      \begin{subfigure}{0.48\textwidth}
        \centering
        \includegraphics[width=\textwidth]{build/leck_turbo_100nbar.pdf}
        \caption{Gleichgewichtsdruck $p_\text{G} = \SI{100}{\nano\bar}$.}
        \label{fig:turbo_leck_100}
      \end{subfigure}
      \hfill
      \begin{subfigure}{0.48\textwidth}
        \centering
        \includegraphics[width=\textwidth]{build/leck_turbo_200nbar.pdf}
        \caption{}Gleichgewichtsdruck $p_\text{G} = \SI{200}{\nano\bar}$.
        \label{fig:turbo_leck_200}
      \end{subfigure}
      \caption{Die gemittelten Drücke der Messungen zu den verschiedenen Gleichgewichtsdrücken mit der jeweiligen linearen Ausgleichsrechnung.}
      \label{fig:turbo_leck}
    \end{figure}

    \noindent Durch Vergleich mit der Gleichung \eqref(??) folgt, dass das Saugvermögen $S$ sich hier durch
    \begin{equation*}
      S = \frac{m \cdot V}{p_\text{G}}
    \end{equation*}
    aus der Steigung der linearen Ausgleichsrechnung berechnet. Es werden folgenden Werte berechnet:
    \begin{align*}
      \text{für} \, p_\text{G} &= \SI{50}{\nano\bar} & \text{für} \, p_\text{G} &= \SI{70}{\nano\bar} \\
      S &= \SI{7.4060 \pm 2.3631}{\litre\per\second}  & S &= \SI{9.0885 \pm 2.903}{\litre\per\second}  \\
      \\
      \text{für} \, p_\text{G} &= \SI{100}{\nano\bar} & \text{für} \, p_\text{G} &= \SI{200}{\nano\bar} \\
      S &= \SI{17.7988 \pm 5.6696}{\litre\per\second}  & S &= \SI{31.6721 \pm 10.0229}{\litre\per\second}  
    \end{align*}

  \subsubsection{Evakuierungskurve}

    \noindent Bei der Messung der Evakuierungskurve der Turbomolekularpumpe wird der Tank mit Luft befüllt bis zu einem Druck von $p_0 = \SI{5e-3}{\milli\bar}$. Dann wird der Rezipient 
    abgedichtet und in Abständen von $\increment t = \SI{10}{\second}$ wird der Druck abgelesen, bis $ t = \SI{120}{\second}$ erreicht ist. Die Messung wurde drei mal ausgeführt.
    In der weiteren Auswertung wird eine lineare Ausgleichsrechnung gemacht, wobei ein linearer Zusammenhang zwischen dem Logarithmusausdruck 
    \begin{equation*}
      \ln(F) = \ln \left( \frac{p(t) = p_\text{E}}{p_0 - p_\text{E}}\right) = m \cdot t + n \, 
    \end{equation*} 
    und der Zeit, wie es in \eqref{} hergeleitet wird. Die Messdaten der einzelnen Messungen sind neben dem jeweiligen Logarithmusausdruck in der \autoref{tab:turbo_eva} aufgelistet. 
    Der Enddruck der Pumpe wurde zu $p_\text{E} = \SI{10.9 \pm 3.27}{\nano\bar}$ bestimmt. 

    \begin{table}
      \centering
      \caption{Die einzelnen Messdaten der Evakuierungsmessung mit der Turbopumpe. Zusätlich ist jeweils noch der Ausdruck $\ln(F)$ aufgelistet, wobei $F$ der Quotient $F = \frac{p(t) - p_\text{E}}{p_0 - p_\text{E}}$ ist. }
      \label{tab:turbo_eva}
      \sisetup{table-format=4.2}
      \begin{center}
        \addtolength{\leftskip} {-4.5cm} % increase (absolute) value if needed
        \addtolength{\rightskip}{-4.5cm}
      \begin{tabular*}{1.4\textwidth}{@{\extracolsep{\fill}} S[table-format=3.0] S @{${}\pm{}$} S S[table-format=2.2] @{${}\pm{}$} S[table-format=1.2]
                                          S @{${}\pm{}$} S S[table-format=2.2] @{${}\pm{}$} S[table-format=1.2]
                                          S @{${}\pm{}$} S S[table-format=2.2] @{${}\pm{}$} S[table-format=1.2]}
        \toprule
        & \multicolumn{4}{c}{Messung 1} & \multicolumn{4}{c}{Messung 2} & \multicolumn{4}{c}{Messung 3} \\
        \cmidrule(lr){2-5} \cmidrule(lr){6-9} \cmidrule(lr){10-13}
        {$t \mathbin{/} \si{\second}$} & \multicolumn{2}{c}{$p \mathbin{/} \si{\nano\bar}$} & \multicolumn{2}{c}{$\ln(F)$} & \multicolumn{2}{c}{$p \mathbin{/} \si{\nano\bar}$} & \multicolumn{2}{c}{$\ln(F)$} & \multicolumn{2}{c}{$p \mathbin{/} \si{\nano\bar}$} & \multicolumn{2}{c}{$\ln(F)$} \\
          0 & 5000.00 &1500.00 &  0.00 & 0.42 &5000.00 & 1500.00 &  0.00 & 0.42 & 5000.00 & 1500.00 &  0.00 & 0.42 \\
         10 &  194.00 &  58.20 & -3.31 & 0.43 & 176.00 &   52.80 & -3.40 & 0.43 &  202.00 &   60.60 & -3.26 & 0.43 \\
         20 &   44.60 &  13.38 & -4.99 & 0.50 &  35.30 &   10.59 & -5.32 & 0.54 &   42.40 &   12.72 & -5.06 & 0.51 \\
         30 &   25.20 &   7.56 & -5.85 & 0.64 &  22.00 &    6.60 & -6.10 & 0.72 &   23.30 &    6.99 & -5.99 & 0.69 \\
         40 &   21.20 &   6.36 & -6.18 & 0.75 &  19.20 &    5.76 & -6.39 & 0.85 &   20.00 &    6.00 & -6.30 & 0.80 \\
         50 &   18.90 &   5.67 & -6.43 & 0.87 &  17.40 &    5.22 & -6.64 & 0.99 &   18.20 &    5.46 & -6.52 & 0.92 \\
         60 &   17.50 &   5.25 & -6.62 & 0.98 &  16.80 &    5.04 & -6.74 & 1.06 &   17.50 &    5.25 & -6.62 & 0.98 \\
         70 &   16.80 &   5.04 & -6.74 & 1.06 &  15.80 &    4.74 & -6.92 & 1.21 &   16.30 &    4.89 & -6.82 & 1.12 \\
         80 &   15.90 &   4.77 & -6.90 & 1.19 &  15.10 &    4.53 & -7.07 & 1.36 &   15.50 &    4.65 & -6.98 & 1.27 \\
         90 &   15.20 &   4.56 & -7.05 & 1.33 &  14.50 &    4.35 & -7.23 & 1.54 &   14.90 &    4.47 & -7.12 & 1.41 \\
        100 &   14.70 &   4.41 & -7.18 & 1.47 &  14.10 &    4.23 & -7.35 & 1.69 &   14.40 &    4.32 & -7.26 & 1.57 \\
        110 &   14.20 &   4.26 & -7.32 & 1.65 &  13.70 &    4.11 & -7.48 & 1.89 &   14.00 &    4.20 & -7.38 & 1.74 \\
        120 &   13.80 &   4.14 & -7.45 & 1.84 &  13.50 &    4.05 & -7.55 & 2.02 &   13.70 &    4.11 & -7.48 & 1.89 \\
        \bottomrule
      \end{tabular*}
    \end{center}
    \end{table}
  

    \noindent Es wird der Logarithmusausdruck $\ln(F)$ gegen die Zeit geplottet für jede einzelne Messung. In den Daten werden lineare Abhängigkeiten gesucht und dann in diesen Bereichen
    eine lineare Ausgleichsrechung der Form 
    \begin{equation*}
      y = m \cdot t + n
    \end{equation*}
    mit Hilfe von \cite{scipy} durchgeführt. Es werden drei Bereiche pro Messung gesucht. \\
    Dieses Vorgehen ist für die erste Messung in der \autoref{fig:evaku_turbo_1} zu sehen, hier werden keine Datenpunkte exkludiert. Das gleiche Vorgehen wird für
    die beiden anderen Messreihen gemacht, dies ist in der \autoref{fig:evaku_turbo_2} für die zweite Messreihe zu sehen und für die Dritte in der \autoref{fig:evaku_turbo_3}. 
    Es wird kein Datenpunkt exkludiert. 
    Die Parameterdaten für die lineare Ausgleichsrechung sind in der \autoref{tab:fitpara_turbo} aufgelistet. \\ 
    

    \begin{figure}
      \centering
      \includegraphics[width=\textwidth]{build/evakuturbo_1.pdf}
      \caption{Die Messdaten der ersten Evakuierungsmessung aufgetragen in einem Logarithmusausdruck gegen die Zeit. Zusätzlich werden lineare Ausgleichsrechungen in linear wirkende Bereiche gelegt.}
      \label{fig:evaku_turbo_1}
    \end{figure}

    \begin{figure}
      \centering
      \includegraphics[width=\textwidth]{build/evakuturbo_2.pdf}
      \caption{Die Messdaten der zweiten Evakuierungsmessung aufgetragen in einem Logarithmusausdruck gegen die Zeit. Zusätzlich werden lineare Ausgleichsrechungen in linear wirkende Bereiche gelegt.}
      \label{fig:evaku_turbo_2}
    \end{figure}

    \begin{figure}
      \centering
      \includegraphics[width=\textwidth]{build/evakuturbo_3.pdf}
      \caption{Die Messdaten der dritten Evakuierungsmessung aufgetragen in einem Logarithmusausdruck gegen die Zeit. Zusätzlich werden lineare Ausgleichsrechungen in linear wirkende Bereiche gelegt.}
      \label{fig:evaku_turbo_3}
    \end{figure}

    % \begin{table}
      % \centering
      % \caption{Die Fitparameter und die daraus errechneten Saugvermögen für die einzelnen Bereiche der Messung.}
      % \label{tab:fitpara_turbo}
      % \sisetup{table-format=1.4}
      % \begin{tabular}{c S[table-format=2.4] @{${}\pm{}$} S S[table-format=2.4] @{${}\pm{}$} S S @{${}\pm{}$} S
                        % S[table-format=2.4] @{${}\pm{}$} S S[table-format=2.4] @{${}\pm{}$} S S @{${}\pm{}$} S
                        % S[table-format=2.4] @{${}\pm{}$} S S[table-format=2.4] @{${}\pm{}$} S S @{${}\pm{}$} S}
        % \toprule
        % & \multicolumn{6}{c}{Messung 1} & \multicolumn{2}{c}{Messung 2} & \multicolumn{2}{c}{Meesung 3}\\
        % \cmidrule(lr){2-7} \cmidrule(lr){8-13} \cmidrule(lr){14-19}
        % {Druckbereich} & \multicolumn{2}{c}{$m \mathbin{/} \si{1\per\second}$} & \multicolumn{2}{c}{$n$} & \multicolumn{2}{c}{$S \mathbin{/} \si{\litre\per\second}$}
                      %  & \multicolumn{2}{c}{$m \mathbin{/} \si{1\per\second}$} & \multicolumn{2}{c}{$n$} & \multicolumn{2}{c}{$S \mathbin{/} \si{\litre\per\second}$}
                      %  & \multicolumn{2}{c}{$m \mathbin{/} \si{1\per\second}$} & \multicolumn{2}{c}{$n$} & \multicolumn{2}{c}{$S \mathbin{/} \si{\litre\per\second}$} \\
        % \midrule
        % $\SI{30}{\nano\bar} \leq p \leq \SI{5000}{\nano\bar}$   & -0.2499 & 0.0465 & -0.2687 & 0.6009 & 8.2459 & 1.7434 & -0.2660 & 0.0432 & -0.2494 & 0.5577 & 8.7787 & 1.6742 & -0.2533 & 0.0421 & -0.2432 & 0.5439 & 8.3573 & 1.6221 \\
        % $\SI{29.9}{\nano\bar} \leq p \leq \SI{17.5}{\nano\bar}$ & -0.0290 & 0.0022 & -4.9961 & 0.0889 & 0.9583 & 0.1198 & -0.0268 & 0.0013 & -5.3131 & 0.0545 & 0.8830 & 0.0987 & -0.0265 & 0.0026 & -5.2174 & 0.1049 & 0.8742 & 0.1218 \\
        % $\SI{17.4}{\nano\bar} \leq p \leq \SI{13.5}{\nano\bar}$ & -0.0139 & 0.0002 & -5.7855 & 0.0227 & 0.4601 & 0.0467 & -0.0137 & 0.0007 & -5.9593 & 0.0641 & 0.4537 & 0.0508 & -0.0141 & 0.0007 & -5.8293 & 0.0609 & 0.4662 & 0.0515 \\
        % \bottomrule
      % \end{tabular}
    % \end{table}

    \begin{table}[h]
      \centering
      \caption{Die Fitparameter und die daraus errechneten Saugvermögen für die einzelnen Bereiche der Messung.}
      \label{tab:fitpara_turbo_alt}
      \sisetup{table-format=1.4}
      \begin{tabular}{c S[table-format=2.4] @{${}\pm{}$} S S[table-format=2.4] @{${}\pm{}$} S S[table-format=2.4] @{${}\pm{}$} S}
        \toprule
        {Bereiche:} & \multicolumn{2}{c}{$\SI{30}{\nano\bar} \leq p \leq \SI{5000}{\nano\bar}$} & \multicolumn{2}{c}{$\SI{29.9}{\nano\bar} \leq p \leq \SI{17.5}{\nano\bar}$} & \multicolumn{2}{c}{$\SI{17.4}{\nano\bar} \leq p \leq \SI{13.5}{\nano\bar}$}\\
        \midrule
        Messung 1 \\ 
        \cmidrule(lr){1-1}
        $m \mathbin{/} \left(\si{1\per\second}\right)$ & -0.2499 & 0.0465 & -0.0290 & 0.0022 & -0.0139 & 0.0002 \\
        $n$                                            & -0.2687 & 0.6009 & -4.9961 & 0.0889 & -5.7855 & 0.0227 \\
        $S \mathbin{/} \left(\si{L\per\second}\right)$ &  8.2459 & 1.7434 &  0.9583 & 0.1198 &  0.4601 & 0.0467 \\
        \midrule
        Messung 2 \\ 
        \cmidrule(lr){1-1}
        $m \mathbin{/} \left(\si{1\per\second}\right)$ & -0.2660 & 0.0432 & -0.0268 & 0.0013 & -0.0137 & 0.0007 \\
        $n$                                            & -0.2494 & 0.5577 & -5.3131 & 0.0545 & -5.9593 & 0.0641 \\
        $S \mathbin{/} \left(\si{L\per\second}\right)$ &  8.7787 & 1.6742 &  0.8830 & 0.0987 &  0.4537 & 0.0508 \\
        \midrule
        Messung 3 \\ 
        \cmidrule(lr){1-1}
        $m \mathbin{/} \left(\si{1\per\second}\right)$ & -0.2533 & 0.0421 & -0.0265 & 0.0026 & -0.0141 & 0.0007 \\
        $n$                                            & -0.2432 & 0.5439 & -5.2174 & 0.1049 & -5.8293 & 0.0609 \\
        $S \mathbin{/} \left(\si{L\per\second}\right)$ &  8.3573 & 1.6221 &  0.8742 & 0.1218 &  0.4662 & 0.0515 \\      
        \bottomrule
      \end{tabular}
    \end{table}

    \noindent Aus der Steigung der Ausgleichsgeraden kann das Saugvermögen berechnet werden, es gilt der Zusammenhang
    \begin{equation*}
      S = - m \cdot V\, .
    \end{equation*}
    Dies folgt aus der \autoref{sec:Theorie} oder soo. Die entsprechenden Saugvermögen sind auch in der \autoref{tab:fitpara_turbo_alt} aufgelistet. 
    Die Fehler der Werte werden mit der Gleichung \eqref{eqn:err_saug_eva} berechnet.\\ 
    Abschließend werden noch die Mittelwerte berechnet, die wie folgt lauten:
    \begin{align*}
      \SI{30}{\nano\bar} \leq &p \leq \SI{5000}{\nano\bar} & S &= \SI{8.4606 \pm 0.1325}{L\per\second} \\
      \SI{29.9}{\nano\bar} \leq &p \leq \SI{17.5}{\nano\bar}& S &= \SI{0.9052 \pm 0.0218}{L\per\second} \\
      \SI{17.4}{\nano\bar} \leq &p \leq \SI{13.5}{\nano\bar} & S &= \SI{0.4560 \pm 0.0029}{L\per\second}
    \end{align*}
    Hier entspricht der Fehler dem Fehler des Mittelwertes.

\subsection{Drehschieberpumpe}

    \noindent Nun werden die gleichen Auswertungsschritte bei den Messwerten der Drehschieberpumpe angewendet. Das Volumen des Rezipienten beträgt nun $V = \SI{34 \pm 3.4}{\litre}$, da die 
    Drehschieberpumpe hinter die dann ausgeschaltete Turbomolekularpumpe geschalten, die Schläuche und die Pumpe, welche nun zusätzlich evakuierten werden müssen, haben also ein Volumen von $V_\text{Zusatz} = \SI{1.0}{\litre}$. 

    \subsubsection{Leckratenmessung}

    \noindent Analog zum Vorgehen bei der Turbomolekularpumpe wird bei laufender Pumpe durch geregeltes Öffnen des Ventils V4 ?? ein Gleichgewichtsdruck eingestellt. Hier werden die
    Gleichgewichtsdrücke $p_\text{G} = \SI{0.5}{\milli\bar}, \, \SI{10}{\milli\bar}, \, \SI{50}{\milli\bar}, \, \SI{100}{\milli\bar}$ benutzt. Für jeden Gleichgewichtsdruck werden 
    drei Messungen durchgeführt. Der Messzeitraum beträgt $\SI{200}{\second}$, es werden in Schritten von $\increment t = \SI{10}{\second}$ Daten aufgenommen. Über die drei aufgenommenen
    Messungen des Drucks wird das arithmetische Mittel genommen. 
    Die Daten befinden sich für den Gleichgewichtsdruck $p_\text{G} = \SI{0.5}{\milli\bar}$ in der \autoref{tab:dreh_leck_05}, und für $p_\text{G} = \SI{10}{\milli\bar}$ in der 
    \autoref{tab:dreh_leck_10}. Für den Gleichgewichtsdruck $p_\text{G} = \SI{50}{\milli\bar}$ sind die Messdaten in der \autoref{tab:dreh_leck_50} und für $p_\text{G} = \SI{100}{\milli\bar}$
    in der \autoref{tab:dreh_leck_100}.

    \begin{table}[h]
      \centering
      \caption{Die Messwerte der einzelnen Messungen und der daraus gemittelte Druckwert für die Leckratenmessung der Drehschieberpumpe mit den 
      Gleichgewichtsdruck $p_\text{G} = \SI{0.5}{\milli\bar}$.}
      \label{tab:dreh_leck_05}
      \sisetup{table-format=4.1}
      \begin{tabular}{S[table-format=3.0] S @{${}\pm{}$} S S @{${}\pm{}$} S S @{${}\pm{}$} S S[table-format=1.4] @{${}\pm{}$} S[table-format=1.4]}
      \toprule
      {$t \mathbin{/} \si{\second} $} & \multicolumn{2}{c}{$p_\text{M1} \mathbin{/} \si{\micro\bar}$} & \multicolumn{2}{c}{$p_\text{M2} \mathbin{/} \si{\micro\bar}$} & \multicolumn{2}{c}{$p_\text{M3} \mathbin{/} \si{\micro\bar}$} & \multicolumn{2}{c}{$p_\text{Mittel} \mathbin{/} \si{\milli\bar}$} \\
      \midrule
        0 &  500.0 & 150.0 & 500.0 & 150.0 & 500.0 & 150.0 & 0.5000 & 0.0000 \\  
       10 &  513.0 & 153.9 & 506.0 & 151.8 & 517.0 & 155.1 & 0.5120 & 0.0026 \\
       20 &  532.0 & 159.6 & 522.0 & 156.6 & 538.0 & 161.4 & 0.5307 & 0.0038 \\
       30 &  552.0 & 165.6 & 539.0 & 161.7 & 563.0 & 168.9 & 0.5513 & 0.0057 \\
       40 &  571.0 & 171.3 & 558.0 & 167.4 & 585.0 & 175.5 & 0.5713 & 0.0064 \\
       50 &  591.0 & 177.3 & 575.0 & 172.5 & 605.0 & 181.5 & 0.5903 & 0.0071 \\
       60 &  608.0 & 182.4 & 592.0 & 177.6 & 627.0 & 188.1 & 0.6090 & 0.0083 \\
       70 &  627.0 & 188.1 & 608.0 & 182.4 & 648.0 & 194.4 & 0.6277 & 0.0094 \\
       80 &  645.0 & 193.5 & 624.0 & 187.2 & 669.0 & 200.7 & 0.6460 & 0.0106 \\
       90 &  664.0 & 199.2 & 640.0 & 192.0 & 691.0 & 207.3 & 0.6650 & 0.0120 \\
      100 &  682.0 & 204.6 & 657.0 & 197.1 & 712.0 & 213.6 & 0.6837 & 0.0130 \\
      110 &  701.0 & 210.3 & 673.0 & 201.9 & 734.0 & 220.2 & 0.7027 & 0.0144 \\
      120 &  719.0 & 215.7 & 689.0 & 206.7 & 758.0 & 227.4 & 0.7220 & 0.0163 \\
      130 &  738.0 & 221.4 & 706.0 & 211.8 & 779.0 & 233.7 & 0.7410 & 0.0172 \\
      140 &  757.0 & 227.1 & 722.0 & 216.6 & 801.0 & 240.3 & 0.7600 & 0.0187 \\
      150 &  776.0 & 232.8 & 739.0 & 221.7 & 822.0 & 246.6 & 0.7790 & 0.0196 \\
      160 &  795.0 & 238.5 & 756.0 & 226.8 & 844.0 & 253.2 & 0.7983 & 0.0208 \\
      170 &  814.0 & 244.2 & 773.0 & 231.9 & 866.0 & 259.8 & 0.8177 & 0.0220 \\
      180 &  833.0 & 249.9 & 790.0 & 237.0 & 889.0 & 266.7 & 0.8373 & 0.0234 \\
      190 &  850.0 & 255.0 & 806.0 & 241.8 & 910.0 & 273.0 & 0.8553 & 0.0246 \\
      200 &  870.0 & 261.0 & 823.0 & 246.9 & 934.0 & 280.2 & 0.8757 & 0.0263 \\
      \bottomrule
      \end{tabular}
    \end{table}

    \begin{table}[h]
      \centering
      \caption{Die Messwerte der einzelnen Messungen und der daraus gemittelte Druckwert für die Leckratenmessung der Drehschieberpumpe mit den 
      Gleichgewichtsdruck $p_\text{G} = \SI{10}{\milli\bar}$.}
      \label{tab:dreh_leck_10}
      \sisetup{table-format=2.1}
      \begin{tabular}{S[table-format=3.0] S @{${}\pm{}$} S[table-format=2.2] S @{${}\pm{}$} S[table-format=2.2] S @{${}\pm{}$} S[table-format=2.2] S[table-format=2.2] @{${}\pm{}$} S[table-format=1.2]}
      \toprule
      {$t \mathbin{/} \si{\second} $} & \multicolumn{2}{c}{$p_\text{M1} \mathbin{/} \si{\milli\bar}$} & \multicolumn{2}{c}{$p_\text{M2} \mathbin{/} \si{\milli\bar}$} & \multicolumn{2}{c}{$p_\text{M3} \mathbin{/} \si{\milli\bar}$} & \multicolumn{2}{c}{$p_\text{Mittel} \mathbin{/} \si{\milli\bar}$} \\
      \midrule
        0 & 10.0 &  3.00 & 10.0 &  3.00 & 10.0 &  3.00 & 10.00 & 0.00 \\  
       10 & 12.7 &  3.81 & 12.6 &  3.78 & 12.9 &  3.87 & 12.73 & 0.07 \\
       20 & 14.9 &  4.47 & 14.8 &  4.44 & 15.0 &  4.50 & 14.90 & 0.05 \\
       30 & 17.0 &  5.10 & 17.0 &  5.10 & 17.2 &  5.16 & 17.07 & 0.05 \\
       40 & 19.1 &  5.73 & 19.1 &  5.73 & 19.2 &  5.76 & 19.13 & 0.03 \\
       50 & 21.3 &  6.39 & 21.5 &  6.45 & 21.6 &  6.48 & 21.47 & 0.07 \\
       60 & 24.0 &  7.20 & 24.1 &  7.23 & 24.3 &  7.29 & 24.13 & 0.07 \\
       70 & 26.6 &  7.98 & 26.7 &  8.01 & 26.8 &  8.04 & 26.70 & 0.04 \\
       80 & 29.1 &  8.73 & 29.1 &  8.73 & 29.2 &  8.76 & 29.13 & 0.03 \\
       90 & 31.3 &  9.39 & 31.3 &  9.39 & 31.4 &  9.42 & 31.33 & 0.03 \\
      100 & 33.5 & 10.05 & 33.9 & 10.17 & 33.5 & 10.05 & 33.63 & 0.11 \\
      110 & 35.3 & 10.59 & 35.3 & 10.59 & 35.4 & 10.62 & 35.33 & 0.03 \\
      120 & 37.2 & 11.16 & 37.4 & 11.22 & 37.4 & 11.22 & 37.33 & 0.05 \\
      130 & 39.1 & 11.73 & 39.0 & 11.70 & 39.2 & 11.76 & 39.10 & 0.05 \\
      140 & 41.4 & 12.42 & 41.4 & 12.42 & 41.5 & 12.45 & 41.43 & 0.03 \\
      150 & 44.4 & 13.32 & 44.4 & 13.32 & 44.4 & 13.32 & 44.40 & 0.00 \\
      160 & 47.0 & 14.10 & 47.3 & 14.19 & 47.2 & 14.16 & 47.17 & 0.07 \\
      170 & 49.6 & 14.88 & 49.8 & 14.94 & 49.8 & 14.94 & 49.73 & 0.05 \\
      180 & 52.1 & 15.63 & 51.9 & 15.57 & 52.2 & 15.66 & 52.06 & 0.07 \\
      190 & 54.6 & 16.38 & 54.5 & 16.35 & 54.7 & 16.41 & 54.60 & 0.05 \\
      200 & 56.9 & 17.07 & 56.8 & 17.04 & 56.9 & 17.07 & 56.87 & 0.03 \\
      \bottomrule
      \end{tabular}
    \end{table}

    \begin{table}[h]
      \centering
      \caption{Die Messwerte der einzelnen Messungen und der daraus gemittelte Druckwert für die Leckratenmessung der Drehschieberpumpe mit den 
      Gleichgewichtsdruck $p_\text{G} = \SI{50}{\milli\bar}$.}
      \label{tab:dreh_leck_50}
      \sisetup{table-format=4.1}
      \begin{tabular}{S[table-format=3.0] S @{${}\pm{}$} S[table-format=3.2] S @{${}\pm{}$} S[table-format=3.2] S @{${}\pm{}$} S[table-format=3.2] S[table-format=3.2] @{${}\pm{}$} S[table-format=2.2]}
      \toprule
      {$t \mathbin{/} \si{\second} $} & \multicolumn{2}{c}{$p_\text{M1} \mathbin{/} \si{\milli\bar}$} & \multicolumn{2}{c}{$p_\text{M2} \mathbin{/} \si{\milli\bar}$} & \multicolumn{2}{c}{$p_\text{M3} \mathbin{/} \si{\milli\bar}$} & \multicolumn{2}{c}{$p_\text{Mittel} \mathbin{/} \si{\milli\bar}$} \\
      \midrule
        0 &   50.0 &  15.00 &   50.0 &  15.00 &  50.0 &  15.00 &  50.00 &  0.00 \\  
       10 &   68.9 &  20.67 &   69.5 &  20.85 &  66.8 &  20.04 &  68.40 &  0.67 \\
       20 &   87.5 &  26.25 &   86.8 &  26.04 &  84.5 &  25.35 &  86.27 &  0.74 \\
       30 &  109.0 &  54.50 &  108.0 &  54.00 & 102.0 &  51.00 & 106.33 &  1.78 \\
       40 &  135.0 &  67.50 &  137.0 &  68.50 & 130.0 &  65.00 & 134.00 &  1.70 \\
       50 &  161.0 &  80.50 &  163.0 &  81.50 & 155.0 &  77.50 & 159.67 &  1.96 \\
       60 &  190.0 &  95.00 &  189.0 &  94.50 & 181.0 &  90.50 & 186.67 &  2.33 \\
       70 &  221.0 & 110.50 &  218.0 & 109.00 & 208.0 & 104.00 & 215.67 &  3.29 \\
       80 &  264.0 & 132.00 &  266.0 & 133.00 & 246.0 & 123.00 & 258.67 &  5.19 \\
       90 &  297.0 & 148.50 &  302.0 & 151.00 & 284.0 & 142.00 & 294.33 &  4.38 \\
      100 &  342.0 & 171.00 &  338.0 & 169.00 & 325.0 & 162.50 & 335.00 &  4.19 \\
      110 &  374.0 & 187.00 &  379.0 & 189.50 & 369.0 & 184.50 & 374.00 &  2.36 \\
      120 &  431.0 & 215.50 &  435.0 & 217.50 & 395.0 & 197.50 & 420.33 & 10.36 \\
      130 &  502.0 & 251.00 &  498.0 & 249.00 & 455.0 & 227.50 & 485.00 & 12.28 \\
      140 &  560.0 & 280.00 &  569.0 & 284.50 & 526.0 & 263.00 & 551.67 & 10.69 \\
      150 &  642.0 & 321.00 &  649.0 & 324.50 & 582.0 & 291.00 & 624.33 & 17.36 \\
      160 &  710.0 & 355.00 &  704.0 & 352.00 & 688.0 & 344.00 & 700.67 &  5.36 \\
      170 &  795.0 & 397.50 &  808.0 & 404.00 & 740.0 & 370.00 & 781.00 & 17.02 \\
      180 &  874.0 & 437.00 &  880.0 & 440.00 & 820.0 & 410.00 & 858.00 & 15.58 \\
      190 &  945.0 & 472.50 & 1000.0 & 500.00 & 895.0 & 447.50 & 946.67 & 24.76 \\
      200 & 1000.0 & 500.00 & 1000.0 & 500.00 & 965.0 & 482.50 & 988.33 &  9.53 \\
      \bottomrule
      \end{tabular}
    \end{table}

    \begin{table}[h]
      \centering
      \caption{Die Messwerte der einzelnen Messungen und der daraus gemittelte Druckwert für die Leckratenmessung der Drehschieberpumpe mit den 
      Gleichgewichtsdruck $p_\text{G} = \SI{100}{\milli\bar}$.}
      \label{tab:dreh_leck_100}
      \sisetup{table-format=4.1}
      \begin{tabular}{S[table-format=3.0] S @{${}\pm{}$} S[table-format=3.2] S @{${}\pm{}$} S[table-format=3.2] S @{${}\pm{}$} S[table-format=3.2] S[table-format=4.2] @{${}\pm{}$} S[table-format=2.2]}
      \toprule
      {$t \mathbin{/} \si{\second} $} & \multicolumn{2}{c}{$p_\text{M1} \mathbin{/} \si{\milli\bar}$} & \multicolumn{2}{c}{$p_\text{M2} \mathbin{/} \si{\milli\bar}$} & \multicolumn{2}{c}{$p_\text{M3} \mathbin{/} \si{\milli\bar}$} & \multicolumn{2}{c}{$p_\text{Mittel} \mathbin{/} \si{\milli\bar}$} \\
      \midrule
        0 &  100.0 &  30.0 &  100.0 &  30.0 &  100.0 &  30.0 &  100.00 &  0.00 \\  
       10 &  138.0 &  69.0 &  156.0 &  78.0 &  158.0 &  79.0 &  150.67 &  5.19 \\
       20 &  186.0 &  93.0 &  206.0 & 103.0 &  208.0 & 104.0 &  200.00 &  5.73 \\
       30 &  253.0 & 126.5 &  281.0 & 140.5 &  291.0 & 145.5 &  275.00 &  9.29 \\
       40 &  323.0 & 161.5 &  349.0 & 174.5 &  356.0 & 178.0 &  342.67 &  8.20 \\
       50 &  397.0 & 198.5 &  439.0 & 219.5 &  448.0 & 224.0 &  428.00 & 12.83 \\
       60 &  512.0 & 256.0 &  556.0 & 278.0 &  577.0 & 288.5 &  548.33 & 15.64 \\
       70 &  655.0 & 327.5 &  683.0 & 341.5 &  694.0 & 347.0 &  677.33 &  9.48 \\
       80 &  788.0 & 394.0 &  882.0 & 441.0 &  876.0 & 438.0 &  848.67 & 24.81 \\
       90 &  979.0 & 489.5 & 1000.0 & 500.0 & 1000.0 & 500.0 &  993.00 &  5.72 \\
      100 & 1000.0 & 500.0 & 1000.0 & 500.0 & 1000.0 & 500.0 & 1000.00 &  0.00 \\
      110 & 1000.0 & 500.0 & 1000.0 & 500.0 & 1000.0 & 500.0 & 1000.00 &  0.00 \\
      120 & 1000.0 & 500.0 & 1000.0 & 500.0 & 1000.0 & 500.0 & 1000.00 &  0.00 \\
      130 & 1000.0 & 500.0 & 1000.0 & 500.0 & 1000.0 & 500.0 & 1000.00 &  0.00 \\
      140 & 1000.0 & 500.0 & 1000.0 & 500.0 & 1000.0 & 500.0 & 1000.00 &  0.00 \\
      150 & 1000.0 & 500.0 & 1000.0 & 500.0 & 1000.0 & 500.0 & 1000.00 &  0.00 \\
      160 & 1000.0 & 500.0 & 1000.0 & 500.0 & 1000.0 & 500.0 & 1000.00 &  0.00 \\
      170 & 1000.0 & 500.0 & 1000.0 & 500.0 & 1000.0 & 500.0 & 1000.00 &  0.00 \\
      180 & 1000.0 & 500.0 & 1000.0 & 500.0 & 1000.0 & 500.0 & 1000.00 &  0.00 \\
      190 & 1000.0 & 500.0 & 1000.0 & 500.0 & 1000.0 & 500.0 & 1000.00 &  0.00 \\
      200 & 1000.0 & 500.0 & 1000.0 & 500.0 & 1000.0 & 500.0 & 1000.00 &  0.00 \\
      \bottomrule
      \end{tabular}
    \end{table}

    \noindent Die gemittelten Drücke werden gegen die Zeit aufgetragen. Dies ist in den Abbildungen in der \autoref{fig:dreh_leck} zu sehen. Dazu sind in lineare Ausgleichsrechnungen 
    mit \cite{scipy} durchgeführt worden. Die Fitparameter ergeben sich zu:
    \begin{align*}
      \text{Für} \,  p_\text{G} &= \SI{0.5}{\milli\bar} & \text{Für} \,  p_\text{G} &= \SI{10}{\milli\bar}\\
      m &= \SI{1.8965(5)e-3}{\milli\bar\per\second} & m &= \SI{0.2313 \pm 0.0018}{\milli\bar\per\second} \\
      n &= \SI{0.4949 \pm 0.0006}{\milli\bar} & n &= \SI{10.1192 \pm 0.2055}{\milli\bar} \\
      \\
      \text{Für} \,  p_\text{G} &= \SI{50}{\milli\bar} & \text{Für} \,  p_\text{G} &= \SI{100}{\milli\bar}\\
      m &= \SI{4.7505 \pm 0.2513}{\milli\bar\per\second} & m &= \SI{4.9360 \pm 0.5869}{\milli\bar\per\second} \\
      n &= \SI{-64.3325 \pm 29.3767}{\milli\bar} & n &= \SI{247.5253 \pm 68.6133}{\milli\bar} 
    \end{align*}
    Es werden keine Datenpunkte exkludiert. \\
    Aus dem Steigungen der linearen Ausgleichsrechnung werden die Saugvermögen berechnet:
    \begin{align*}
      \text{für} \, p_\text{G} &= \SI{0.5}{\milli\bar} & \text{für} \, p_\text{G} &= \SI{10}{\milli\bar} \\
      S &= \SI{0.1290 \pm 0.0408}{\litre\per\second}  & S &= \SI{0.7865 \pm 0.2488}{\litre\per\second}  \\
      \\
      \text{für} \, p_\text{G} &= \SI{50}{\milli\bar} & \text{für} \, p_\text{G} &= \SI{100}{\milli\bar} \\
      S &= \SI{3.2303 \pm 1.0357}{\litre\per\second}  & S &= \SI{1.6782 \pm 0.5670}{\litre\per\second}  
    \end{align*}

    \begin{figure}[h]
      \begin{subfigure}{0.48\textwidth}
        \centering
        \includegraphics[width=\textwidth]{build/leck_dreh_05mbar.pdf}
        \caption{Gleichgewichtsdruck $p_\text{G} = \SI{0.5}{\milli\bar}$.}
        \label{fig:dreh_leck_05}
      \end{subfigure}
      \hfill
      \begin{subfigure}{0.48\textwidth}
        \centering
        \includegraphics[width=\textwidth]{build/leck_dreh_10mbar.pdf}
        \caption{Gleichgewichtsdruck $p_\text{G} = \SI{10}{\milli\bar}$.}
        \label{fig:dreh_leck_10}
      \end{subfigure}
      \hfill
      \begin{subfigure}{0.48\textwidth}
        \centering
        \includegraphics[width=\textwidth]{build/leck_dreh_50mbar.pdf}
        \caption{Gleichgewichtsdruck $p_\text{G} = \SI{50}{\milli\bar}$.}
        \label{fig:dreh_leck_50}
      \end{subfigure}
      \hfill
      \begin{subfigure}{0.48\textwidth}
        \centering
        \includegraphics[width=\textwidth]{build/leck_dreh_100mbar.pdf}
        \caption{}Gleichgewichtsdruck $p_\text{G} = \SI{100}{\milli\bar}$.
        \label{fig:dreh_leck_100}
      \end{subfigure}
      \caption{Die gemittelten Drücke der Messungen zu den verschiedenen Gleichgewichtsdrücken mit der jeweiligen linearen Ausgleichsrechnung.}
      \label{fig:dreh_leck}
    \end{figure}

  \subsubsection{Evakuierungskurve}

    \noindent Die Drehschieberpumpe hat einen Enddruck von $p_\text{E} = \SI{3.85(115)e-3}{\milli\bar}$. Der Rezipient wird für die Aufnahme der Evakuierungskurve mit der Drehschieberpumpe 
    bis zu einem Druck von $p_0 = \SI{1000}{\milli\bar}$ belüftet. Dann wird die Luftzufuhr abgeschiebert und der Druck in Abhängigkeit der Zeit in 
    Schritten von $\increment t = \SI{10}{\second}$ aufgenommen bist $t = \SI{600}{\second}$ erreicht. Die aufgenommenen Messdaten sind in der \autoref{tab:dreh_eva} zu finden. 
    Dort sind außerdem die Ergbnisse der Rechnung 
    \begin{equation*}
      \ln(F) = \ln \left( \frac{p(t) = p_\text{E}}{p_0 - p_\text{E}}\right)
    \end{equation*}
    eingetragen, da diese für jeden Messung einzeln in ein $t - \ln(F)$ Diagramm eingetragen werden. Die Fehler dieses Ausdruckes werden mit der Gleichung \eqref{eqn:err_ln} berechnet.  

    \begin{table}[h]
      \centering
      \caption{Die einzelnen Messdaten der Evakuierungsmessung mit der Drehschieberpumpe. Zusätlich ist jeweils noch der Ausdruck $\ln(F)$ aufgelistet, wobei $F$ der Quotient $F = \frac{p(t) - p_\text{E}}{p_0 - p_\text{E}}$ ist. }
      \label{tab:dreh_eva}
      \sisetup{table-format=4.2}
      \begin{center}
        \addtolength{\leftskip} {-2.5cm} % increase (absolute) value if needed
        \addtolength{\rightskip}{-2.5cm}
      \begin{tabular*}{1.2\textwidth}{@{\extracolsep{\fill}} S[table-format=3.0] S @{${}\pm{}$} S S[table-format=2.2] @{${}\pm{}$} S[table-format=1.2]
                                                                                 S @{${}\pm{}$} S S[table-format=2.2] @{${}\pm{}$} S[table-format=1.2]
                                                                                 S @{${}\pm{}$} S S[table-format=2.2] @{${}\pm{}$} S[table-format=1.2]}
        \toprule
        & \multicolumn{4}{c}{Messung 1} & \multicolumn{4}{c}{Messung 2} & \multicolumn{4}{c}{Messung 3} \\
        \cmidrule(lr){2-5} \cmidrule(lr){6-9} \cmidrule(lr){10-13}
        {$t \mathbin{/} \si{\second}$} & \multicolumn{2}{c}{$p \mathbin{/} \si{\nano\bar}$} & \multicolumn{2}{c}{$\ln(F)$} & \multicolumn{2}{c}{$p \mathbin{/} \si{\nano\bar}$} & \multicolumn{2}{c}{$\ln(F)$} & \multicolumn{2}{c}{$p \mathbin{/} \si{\nano\bar}$} & \multicolumn{2}{c}{$\ln(F)$} \\
          0 &1000.00 & 500.00 &  0.00 &  0.70 &1000.00 & 500.00 &  0.00 &  0.70 & 1000.00 & 500.00 &  0.00 & 0.70 \\
         10 & 798.00 & 399.00 & -0.22 &  0.70 & 800.00 & 400.00 & -0.22 &  0.70 & 1000.00 & 500.00 &  0.00 & 0.70 \\
         20 & 442.00 & 221.00 & -0.81 &  0.70 & 450.00 & 225.00 & -0.79 &  0.70 & 1000.00 & 500.00 &  0.00 & 0.70 \\
         30 & 278.00 & 139.00 & -1.28 &  0.70 & 282.00 & 141.00 & -1.26 &  0.70 &  624.00 & 312.00 & -0.47 & 0.70 \\
         40 & 172.00 &  86.00 & -1.76 &  0.70 & 177.00 &  88.50 & -1.73 &  0.70 &  359.00 & 179.50 & -1.02 & 0.70 \\
         50 & 111.00 &  55.50 & -2.19 &  0.70 & 119.00 &  59.50 & -2.12 &  0.70 &  218.00 & 109.00 & -1.52 & 0.70 \\
         60 &  77.10 &  23.13 & -2.56 &  0.58 &  79.70 &  23.91 & -2.52 &  0.58 &  142.00 &  71.00 & -1.95 & 0.70 \\
         70 &  53.20 &  15.96 & -2.93 &  0.58 &  55.80 &  16.74 & -2.88 &  0.58 &   93.80 &  28.14 & -2.36 & 0.58 \\
         80 &  38.20 &  11.46 & -3.26 &  0.58 &  40.70 &  12.21 & -3.20 &  0.58 &   65.50 &  19.65 & -2.72 & 0.58 \\
         90 &  29.80 &   8.94 & -3.51 &  0.58 &  31.40 &   9.42 & -3.46 &  0.58 &   45.90 &  13.77 & -3.08 & 0.58 \\
        100 &  22.10 &   6.63 & -3.81 &  0.58 &  23.40 &   7.02 & -3.75 &  0.58 &   34.80 &  10.44 & -3.35 & 0.58 \\
        110 &  17.00 &   5.10 & -4.07 &  0.58 &  17.70 &   5.31 & -4.03 &  0.58 &   26.00 &   7.80 & -3.64 & 0.58 \\
        120 &  13.30 &   3.99 & -4.32 &  0.58 &  14.00 &   4.20 & -4.26 &  0.58 &   19.40 &   5.82 & -3.94 & 0.58 \\
        130 &  10.30 &   3.09 & -4.57 &  0.58 &  10.80 &   3.24 & -4.52 &  0.58 &   15.60 &   4.68 & -4.16 & 0.58 \\
        140 &   8.52 &   2.55 & -4.76 &  0.58 &   8.81 &   2.64 & -4.73 &  0.58 &   12.20 &   3.66 & -4.40 & 0.58 \\
        150 &   7.04 &   2.11 & -4.95 &  0.58 &   7.37 &   2.21 & -4.91 &  0.58 &    9.48 &   2.84 & -4.65 & 0.58 \\
        160 &   6.02 &   1.80 & -5.11 &  0.58 &   6.17 &   1.85 & -5.08 &  0.58 &    7.88 &   2.36 & -4.84 & 0.58 \\
        170 &   5.21 &   1.56 & -5.25 &  0.58 &   5.36 &   1.60 & -5.22 &  0.58 &    6.63 &   1.98 & -5.01 & 0.58 \\
        180 &   4.57 &   1.37 & -5.38 &  0.58 &   4.67 &   1.40 & -5.36 &  0.58 &    5.67 &   1.70 & -5.17 & 0.58 \\
        190 &   4.03 &   1.20 & -5.51 &  0.58 &   4.12 &   1.23 & -5.49 &  0.58 &    4.91 &   1.47 & -5.31 & 0.58 \\
        200 &   3.61 &   1.08 & -5.62 &  0.58 &   3.68 &   1.10 & -5.60 &  0.58 &    4.37 &   1.31 & -5.43 & 0.58 \\
        210 &   3.26 &   0.97 & -5.72 &  0.58 &   3.32 &   0.99 & -5.70 &  0.58 &    3.86 &   1.15 & -5.55 & 0.58 \\
        220 &   2.95 &   0.88 & -5.82 &  0.58 &   3.01 &   0.90 & -5.80 &  0.58 &    3.46 &   1.03 & -5.66 & 0.58 \\
        230 &   2.96 &   0.88 & -5.82 &  0.58 &   2.74 &   0.82 & -5.90 &  0.58 &    3.14 &   0.94 & -5.76 & 0.58 \\
        240 &   2.46 &   0.73 & -6.00 &  0.58 &   2.50 &   0.75 & -5.99 &  0.58 &    2.85 &   0.85 & -5.86 & 0.58 \\
        250 &   2.26 &   0.67 & -6.09 &  0.58 &   2.31 &   0.69 & -6.07 &  0.58 &    2.59 &   0.77 & -5.95 & 0.58 \\
        260 &   2.09 &   0.62 & -6.17 &  0.58 &   2.13 &   0.63 & -6.15 &  0.58 &    2.39 &   0.71 & -6.03 & 0.58 \\
        270 &   1.94 &   0.58 & -6.24 &  0.58 &   1.98 &   0.59 & -6.22 &  0.58 &    2.20 &   0.66 & -6.12 & 0.58 \\
        280 &   1.80 &   0.54 & -6.32 &  0.58 &   1.83 &   0.54 & -6.30 &  0.58 &    2.04 &   0.61 & -6.19 & 0.58 \\
        290 &   1.68 &   0.50 & -6.39 &  0.58 &   1.71 &   0.51 & -6.37 &  0.58 &    1.89 &   0.56 & -6.27 & 0.58 \\
        300 &   1.57 &   0.47 & -6.45 &  0.58 &   1.59 &   0.47 & -6.44 &  0.58 &    1.77 &   0.53 & -6.33 & 0.58 \\
        310 &   1.47 &   0.44 & -6.52 &  0.58 &   1.49 &   0.44 & -6.51 &  0.58 &    1.65 &   0.49 & -6.40 & 0.58 \\
        320 &   1.39 &   0.41 & -6.58 &  0.58 &   1.41 &   0.42 & -6.56 &  0.58 &    1.54 &   0.46 & -6.47 & 0.58 \\
        330 &   1.31 &   0.39 & -6.64 &  0.58 &   1.33 &   0.39 & -6.62 &  0.58 &    1.45 &   0.43 & -6.53 & 0.58 \\
        340 &   1.24 &   0.37 & -6.69 &  0.58 &   1.25 &   0.37 & -6.68 &  0.58 &    1.36 &   0.40 & -6.60 & 0.58 \\
        350 &   1.17 &   0.35 & -6.75 &  0.58 &   1.19 &   0.35 & -6.73 &  0.58 &    1.29 &   0.38 & -6.65 & 0.58 \\
        360 &   1.12 &   0.33 & -6.79 &  0.58 &   1.12 &   0.33 & -6.79 &  0.58 &    1.22 &   0.36 & -6.71 & 0.58 \\
        370 &   1.06 &   0.31 & -6.85 &  0.58 &   1.07 &   0.32 & -6.84 &  0.58 &    1.15 &   0.34 & -6.77 & 0.58 \\
        380 &   1.01 &   0.30 & -6.90 &  0.58 &   1.02 &   0.30 & -6.89 &  0.58 &    1.10 &   0.33 & -6.81 & 0.58 \\
        390 &   0.96 &   0.29 & -6.94 &  0.58 &   0.97 &   0.29 & -6.93 &  0.58 &    1.05 &   0.31 & -6.86 & 0.58 \\
        400 &   0.91 &   0.27 & -6.99 &  0.58 &   0.92 &   0.27 & -6.98 &  0.58 &    0.99 &   0.29 & -6.91 & 0.58 \\
        410 &   0.87 &   0.26 & -7.04 &  0.58 &   0.88 &   0.26 & -7.03 &  0.58 &    0.95 &   0.28 & -6.96 & 0.58 \\
        420 &   0.83 &   0.25 & -7.08 &  0.58 &   0.84 &   0.25 & -7.08 &  0.58 &    0.90 &   0.27 & -7.01 & 0.58 \\
        430 &   0.80 &   0.24 & -7.13 &  0.58 &   0.80 &   0.24 & -7.12 &  0.58 &    0.86 &   0.25 & -7.05 & 0.58 \\
        440 &   0.76 &   0.22 & -7.17 &  0.58 &   0.77 &   0.23 & -7.16 &  0.58 &    0.82 &   0.24 & -7.10 & 0.58 \\
        450 &   0.73 &   0.22 & -7.21 &  0.58 &   0.74 &   0.22 & -7.21 &  0.58 &    0.79 &   0.23 & -7.14 & 0.58 \\
        460 &   0.70 &   0.21 & -7.26 &  0.58 &   0.71 &   0.21 & -7.25 &  0.58 &    0.75 &   0.22 & -7.19 & 0.58 \\
        470 &   0.67 &   0.20 & -7.30 &  0.58 &   0.68 &   0.20 & -7.29 &  0.58 &    0.72 &   0.21 & -7.23 & 0.58 \\
        480 &   0.69 &   0.20 & -7.27 &  0.58 &   0.65 &   0.19 & -7.33 &  0.58 &    0.69 &   0.20 & -7.27 & 0.58 \\
        490 &   0.62 &   0.18 & -7.38 &  0.58 &   0.63 &   0.18 & -7.37 &  0.58 &    0.67 &   0.20 & -7.31 & 0.58 \\
        500 &   0.60 &   0.18 & -7.42 &  0.58 &   0.60 &   0.18 & -7.41 &  0.58 &    0.64 &   0.19 & -7.35 & 0.58 \\
        510 &   0.58 &   0.17 & -7.45 &  0.58 &   0.58 &   0.17 & -7.45 &  0.58 &    0.62 &   0.18 & -7.39 & 0.58 \\
        520 &   0.55 &   0.16 & -7.49 &  0.58 &   0.56 &   0.16 & -7.48 &  0.58 &    0.59 &   0.17 & -7.42 & 0.58 \\
        530 &   0.53 &   0.16 & -7.54 &  0.58 &   0.54 &   0.16 & -7.52 &  0.58 &    0.57 &   0.17 & -7.46 & 0.58 \\
        540 &   0.51 &   0.15 & -7.57 &  0.58 &   0.52 &   0.15 & -7.56 &  0.58 &    0.55 &   0.16 & -7.50 & 0.58 \\
        550 &   0.49 &   0.14 & -7.61 &  0.58 &   0.50 &   0.15 & -7.60 &  0.58 &    0.53 &   0.15 & -7.54 & 0.58 \\
        560 &   0.48 &   0.14 & -7.64 &  0.58 &   0.48 &   0.14 & -7.63 &  0.58 &    0.51 &   0.15 & -7.58 & 0.58 \\
        570 &   0.46 &   0.13 & -7.68 &  0.58 &   0.46 &   0.14 & -7.67 &  0.58 &    0.49 &   0.14 & -7.61 & 0.58 \\
        580 &   0.44 &   0.13 & -7.72 &  0.58 &   0.45 &   0.13 & -7.70 &  0.58 &    0.47 &   0.14 & -7.65 & 0.58 \\
        590 &   0.43 &   0.13 & -7.75 &  0.58 &   0.43 &   0.13 & -7.74 &  0.58 &    0.46 &   0.13 & -7.68 & 0.58 \\
        600 &   0.42 &   0.12 & -7.78 &  0.58 &   0.42 &   0.12 & -7.77 &  0.58 &    0.44 &   0.13 & -7.72 & 0.58 \\
        \bottomrule
      \end{tabular*}
    \end{center}
    \end{table}

    \begin{figure}[h]
      \centering
      \includegraphics[width=\textwidth]{build/evakudreh_1.pdf}
      \caption{Die Messdaten der ersten Evakuierungsmessung der Drehschieberpumpe aufgetragen in einem Logarithmusausdruck gegen die Zeit. Zusätzlich werden lineare Ausgleichsrechungen in linear wirkende Bereiche gelegt.}
      \label{fig:evaku_dreh_1}
    \end{figure}

    \begin{figure}[h]
      \centering
      \includegraphics[width=\textwidth]{build/evakudreh_2.pdf}
      \caption{Die Messdaten der zweiten Evakuierungsmessung der Drehschieberpumpe aufgetragen in einem Logarithmusausdruck gegen die Zeit. Zusätzlich werden lineare Ausgleichsrechungen in linear wirkende Bereiche gelegt.}
      \label{fig:evaku_dreh_2}
    \end{figure}

    \begin{figure}[h]
      \centering
      \includegraphics[width=\textwidth]{build/evakudreh_3.pdf}
      \caption{Die Messdaten der dritten Evakuierungsmessung der Drehschieberpumpe aufgetragen in einem Logarithmusausdruck gegen die Zeit. Zusätzlich werden lineare Ausgleichsrechungen in linear wirkende Bereiche gelegt.}
      \label{fig:evaku_dreh_3}
    \end{figure}

    \noindent Die Abbildungen werden auf lineares Verhalten untersucht, wobei jeweils drei verschiedene Bereiche mit linearem Verhalten identifiziert werden. 
    Dies ist für die Messreihe 1 in der \autoref{fig:evaku_dreh_1}, für die Messreihe 2 in der \autoref{fig:evaku_dreh_2} und für die dritte Messreihe in der 
    \autoref{fig:evaku_dreh_3} zu sehen. Bei der dritten Messreihe werden die ersten beiden Datenpunkte exkludiert, da diese so gut wie gleich dem dritten Punkt sind und sich 
    daher vermuten lässt, dass die Zeitnahme begonnen hat, bevor der Tank komplett abgedichtet war.\\ 
    Die Parameterdaten der linearen Ausgleichsrechnungen sind in der Tabelle \autoref{tab:fitpara_dreh_alt} aufgelistet für die verschiedenen Druckbereichen. Das Saugvermögen berechnet sich 
    aus der Steigung $m$ durch 
    \begin{equation*}
      S = m \cdot V\, .
    \end{equation*}
    Die Werte für das Saugvermögen sind auch in der \autoref{tab:fitpara_dreh_alt} aufgelistet, der Fehler berechnet sich nach der Formel \eqref{eqn:err_saug_leck}. 

    \begin{table}[h]
      \centering
      \caption{Die Fitparameter und die daraus errechneten Saugvermögen für die einzelnen Bereiche der Messungen von der Evakuierungskurve der Drehschieberpumpe.}
      \label{tab:fitpara_dreh_alt}
      \sisetup{table-format=1.4}
      \begin{tabular}{c | S[table-format=2.4] @{${}\pm{}$} S | S[table-format=2.4] @{${}\pm{}$} S | S[table-format=2.4] @{${}\pm{}$} S}
        \toprule
        {Bereiche:} & \multicolumn{2}{c}{$\SI{10}{\milli\bar} \leq p \leq \SI{1000}{\milli\bar}$} & \multicolumn{2}{c}{$\SI{1.5}{\milli\bar} \leq p \leq \SI{10}{\milli\bar}$} & \multicolumn{2}{c}{$\SI{0.4}{\milli\bar} \leq p \leq \SI{1.5}{\milli\bar}$}\\
        \midrule
        Messung 1 \\ 
        \cmidrule(lr){1-1}
        $m \mathbin{/} \left(\si{1\per\second}\right)$ & -0.0371 & 0.0014 & -0.0106 & 0.0004 & -0.0043 & 0.0001 \\
        $n$                                            & -0.1380 & 0.0956 & -3.3976 & 0.0929 & -5.2577 & 0.0282 \\
        $S \mathbin{/} \left(\si{L\per\second}\right)$ &  1.2628 & 0.1344 &  0.3612 & 0.0388 &  0.1458 & 0.0147 \\
        \midrule
        Messung 2 \\ 
        \cmidrule(lr){1-1}
        $m \mathbin{/} \left(\si{1\per\second}\right)$ & -0.0367 & 0.0013 & -0.0108 & 0.0004 & -0.0043 & 0.0001 \\
        $n$                                            & -0.1281 & 0.0901 & -3.3426 & 0.0992 & -5.2454 & 0.0289 \\
        $S \mathbin{/} \left(\si{L\per\second}\right)$ &  1.2475 & 0.1321 &  0.3670 & 0.0397 &  0.1462 & 0.0148 \\
        \midrule
        Messung 3 \\ 
        \cmidrule(lr){1-1}
        $m \mathbin{/} \left(\si{1\per\second}\right)$ & -0.0394 & 0.0015 & -0.0121 & 0.0006 & -0.0044 & 0.0001 \\
        $n$                                            &  0.5661 & 0.1173 & -2.8796 & 0.1314 & -5.1153 & 0.0320 \\
        $S \mathbin{/} \left(\si{L\per\second}\right)$ &  1.3389 & 0.1436 &  0.4125 & 0.0459 &  0.1508 & 0.0153 \\      
        \bottomrule
      \end{tabular}
    \end{table}

    \noindent Es werden über die einzelnen Bereiche der Mittelwert der pro Messung berechneten Saugvermögen berechnet. Diese Werte ergeben sich zu:
    \begin{align*}
      \SI{10}{\nano\bar} \leq &p \leq \SI{1000}{\nano\bar} & S &= \SI{1.2830 \pm 0.0789}{L\per\second} \\
      \SI{1.5}{\nano\bar} \leq &p \leq \SI{10}{\nano\bar}& S &= \SI{0.3802 \pm 0.0240}{L\per\second} \\
      \SI{0.4}{\nano\bar} \leq &p \leq \SI{1.5}{\nano\bar} & S &= \SI{0.1476 \pm 0.0086}{L\per\second}
    \end{align*}
    Hier wird der Fahler aus der Fehlerfortpflanzung nach Gauß berechnet. 