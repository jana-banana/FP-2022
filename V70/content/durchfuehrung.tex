    \section{Aufbau}
        In \autoref{fig:Aufbau} ist der Versuchsaufbau zu sehen.
        P1 beschreibt die Drehschieberpumpe, P2 die Turbomolekularpumpe.
        V1 und V2 zeigen auf die Ventile der entsprechenden Pumpen, mit diesen Ventilen können die Pumpen abgeschoben werden.
        Zum Einstellen des Gleichgewichtsdruck dienen die Ventile V3 und das Dosierventil D, welche unter dem Rezipienten zu finden sind.
        Das Piezo-Pirani-Vakuummeter steht bei M1.
        Weiterhin gibt es zwei Kaltkathoden-Vakuummeter M2 und M3.
        Eins ist dabei an der Turbomolekularpumpe angeschlosssen, das andere hinter dem Dosierventil.

        \begin{figure}[H]
            \centering
            \includegraphics[width=\textwidth]{bilder/Aufbau.png}
            \caption{Der Versuchsaufbau.\cite{anleitung}}
            \label{fig:Aufbau}
        \end{figure}

    \section{Durchführung}
    \label{sec:Durchführung}
        Während des Kolloquiums wird die Drehschieberpumpe gestartet.
        Es bildet sich ein Endruck von $\SI{3.85e-3}{\milli\BAR}$.
        Daraufhin kann die Turbomolekularpumpe angeschaltet werden.
        Der Endruck hier beträgt $\SI{1.09e-5}{\milli\BAR}$.

        \subsection{Turbomolekularpumpe}
            \subsubsection{Leckratenmessung}
                Zunächst wird eine Leckratenmessung durchgeführt.
                Mithilfe des gelben Dosierventils wird ein konstanter Gleichgewichtsdruck im Rezipienten eingestellt.
                Dabei ist das Ventil zur Turbomolekularpumpe geöffnet.
                Nun wird das Ventil verschlossen und die Messung wird gestartet, indem der Druck alle $\SI{10}{\second}$ aufgenommen wird.
                Dieser Messvorgang wird insgesamt drei Mal wiederholt für vier verschiedenen Leckraten.
                Ein Messvorgang beträgt dabei $\SI{120}{\second}$.
                Die Gleichgewichtsdrücke betragen $\SI{2e-4}{\milli\BAR}, \SI{1e-4}{\milli\BAR}, \SI{7e-5}{\milli\BAR}$ und $\SI{5e-5}{\milli\BAR}$.

            \subsubsection{Evakuierungskurve}
                Der Rezipient wird über das gelbe Dosierventil bis zu einem Anfangsdruck von $\SI{5e-3}{\milli\BAR}$ gelüftet.
                Dann wird das Ventil schnell geschlossen und die Messung gestartet.
                Auch hier wird der Druck alle $\SI{10}{\second}$ aufgenommen, sodass eine Messung $\SI{120}{\second}$ dauert.
                Dieser Messvorgang wird insgesamt drei Mal durchlaufen.

        \subsection{Drehschieberpumpe}
            Die Messungen zur Drehschieberpumpe laufen analog zu der Messung mit der Turbomolekularpumpe ab.
            \subsubsection{Leckratenmessung}
                Analog wird ein Gleichgewichtsdruck mithilfe des gelben Dosierventiels eingestellt und danach wird die Drehschieberpumpe abgeschoben.
                Der Druck wird alle $\SI{10}{\second}$ aufgenommen, die Messzeit insgesamt beträgt $\SI{200}{\second}$.
                Die Messung wird für vier verschiedene Gleichgewichtsdrücke jeweils drei Mal durchgeführt.
                Die verwendeten Gleichgewichtsdrücke sind $\SI{100}{\milli\BAR}, \SI{50}{\milli\BAR}, \SI{10}{\milli\BAR}$ und $\SI{0.5}{\milli\BAR}$.
            \subsubsection{Evakuierungskurve}
                Der Rezipient wird bei laufender Drehschieberpumpe über das gelbe Dosierventil belüftet, sodass ein Anfangsdruck von $\SI{1000}{\milli\BAR}$ entsteht.
                Das Ventil wird geschlossen und die Messung gestartet.
                Der Druck wird dann alle $\SI{10}{\second}$ aufgenommen, insgesamtfür $\SI{600}{\second}$.
                Dieser Messvorgang wird insgesamt drei Mal durchgeführt.