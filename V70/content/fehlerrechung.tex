\section{Fehlerrechnung}
\label{sec:fehler}

\noindent Bei der Mittelung von $n$ unabhängig gemessenen Werten $x_i$ nach der Gleichung 
\begin{equation*}
    \bar{x} = \frac{1}{n} \sum_{i=1}^n x_i
\end{equation*}
berechnet sich der Fehler des arithmetischen Mittelwertes nach
\begin{equation}
    \sigma_{\bar{x}} = \sqrt{\frac{1}{n(n-1)}\sum_{i=1}^n \left( \bar{x} - x_i \right)^2}\, .
    \label{eqn:err_arth_middel}
\end{equation} 
Bei Rechnungen mit mehreren fehlerbehafteten Größen ist die Gauß'sche Fehlerfortpflanzung zu betrachten. Hierbei ist der Fehler der Funktion $f(x_i)$ 
für unkorrelierte $x_i$  durch
\begin{equation}
    \sigma_f = \sqrt{\sum_{i=1}^n \left(\frac{\partial f}{\partial x_i} \sigma_{x_i}\right)^2}
    \label{eqn:gauss}
\end{equation}
gegeben.
Bei der Mittelung werden statistische Fehler ausgerechnet, welche mit steigender Anzahl $n$ an Messungen kleiner werden, welches mathematisch durch den Bruch $\frac{1}{n(n-1)}$ begründet wird. 
Wird eine Messgröße nur mit einem Messgerät gemessen, welches seine natürliche Messunsicherheit hat, dann handelt es sich um einen systematischen Fehler. 
Der relative Fehler verändert sich nicht bei der Mittelung, die Unsicherheit lässt sich nicht über eine größere Anzahl an Messungen verkleinern.\\
Die Genauigkeit der abgelesenen Druckwerte ist druckabhängig, das Messgerät M2, ein kombinierte Pirani/Kaltkathoden-Sensor der Firma Pfeiffer Vacuum, hat im Druckbereich $ \SI{1e-8}{\milli\bar} \leq p \leq \SI{100}{\milli\bar}$ eine Messunsicherheit von 
$\SI{30}{\percent}$ des Messwertes und im Bereich $\SI{100}{\milli\bar} \leq p \leq \SI{1000}{\milli\bar}$ eine Messunsicherheit von $\SI{50}{\percent}$ des Messwertes. 

\noindent Wird die Gaußsche Fehlerfortpflanzung auf den Logarithmus Ausdruck $\ln\left(\frac{p(t) - p_\text{E}}{p_0 - p_\text{E}}\right)$ angewendet, 
ergibt sich der Fehler 
\begin{equation}
    \sigma_{\ln} =\sqrt{\frac{\sigma^2_{p(t)}}{\left(p(t) - p_\text{E}\right)^2} + \frac{\sigma^2_{p_0}}{\left(p_0 - p_\text{E}\right)^2} 
                    + \sigma^2_{p_\text{E}} \left(\frac{1}{p_0 - p_\text{E}} - \frac{1}{p(t) - p_\text{E}}\right)^2 }\, , 
    \label{eqn:err_ln}
\end{equation}
wobei $\sigma_x$ die Messunsicherheit der Größe $x$ bezeichnet.\\
Die in dem Versuch zu ermittelnden Saugvermögen werden jeweils unter anderen aus der Steigung $m_i$ des $i$-ten linearen Fits berechnet. 
Der lineare Fit wird mithilfe von Scientific python \cite{numpy} angewendet, die Fehler der Parameter werden automatisch analog zur linearen Regression berechnet. 
Damit ergeben sich für die folgenden Ausdrücke die folgenden Messunsicherheiten:
\begin{align}
    S &= - m \cdot V & \rightarrow \sigma_S &= \sqrt{V^2\cdot \sigma_m^2 + m^2 \cdot \sigma_V^2} \label{eqn:err_saug_eva}\\
    S &= \frac{m\cdot V}{p_\text{G}} & \rightarrow \sigma_S &= \sqrt{\frac{V^2}{p_\text{G}^2} \sigma_m^2 + \frac{m^2}{p_\text{G}^2} \sigma_V^2 + \frac{m^2V^2}{p_\text{G}^4} \sigma_{p_{\text{G}}}^2 } \label{eqn:err_saug_leck}
\end{align}
% Bei der Mittelung von $n$ fehlerbehafteten Größen $x_i$ folgt aus der Fehlerfortpflanzung
% \begin{equation}
%     \sigma_{\bar{x}} = \frac{1}{n} \sqrt{ \sum_i^n\sigma_{x_i}^2 }\, .
%     \label{eqn:err_anderer_fehler}
% \end{equation}