\section{Diskussion}
\label{sec:Diskussion}

\noindent Zur Identifizierung der Materialien, aus denen die Würfel bestehen können, werden die Literaturwerte für mögliche Materialien in der 
\autoref{tab:lit} aufgelistet. 

\begin{table}
    \centering
    \caption{Die Literaturwerte des Massenschwächungskoeffizienten $\sigma$ \cite{massenbumms}, der Stoffdichte $\rho$ \cite{dichten} und dem Absorptionskoeffizienten $\mu$ der mögliche Materialien.
    Für Holz werden die Werte aus \cite{Saritha_2015} entnommen.}
    \label{tab:lit}
    \begin{tabular}{c c c c} %S[table-format=2.3] S[table-format=2.2] S[table-format=1.3]}
        \toprule
        {Stoff} & {$ \sigma \mathbin{/}  \left(\SI{e-2}{\centi\metre\squared\per\gram}\right)$} & {$\rho \mathbin{/}  \left(\si{\gram\per\centi\metre\tothe{3}}\right)$} & {$\mu \mathbin{/} \left( \si{\per\centi\metre\cubic}\right)$} \\
        \midrule
        Aluminium & 7,802   & 2,7   & 0,211 \\
        Blei      & 12,48   & 11,34 & 1,415 \\
        Eisen     & 7,704   & 7,87  & 0,606 \\
        Messing   & 7,651   & 8,4   & 0,642 \\
        Holz      & 8,0     & 0,3 - 1,2  & 0,024 - 0,096 \\
        \bottomrule
    \end{tabular}
\end{table}

\noindent Die beiden homogenen Würfel 2 und 3 werden jeweils nur mit weniger Projektionen gemessen. Zur Identifizierung der Materialien werden die 
experimentell ermittelten Absorptionskoeffizienten mit den Werten aus der \autoref{tab:lit} verglichen. Der Absorptionskoeffizient des Würfel 2 wird zu
\begin{equation*}
    \mu_{\text{Würfel 2}} = \SI{0.105(6)}{\per\centi\metre}
\end{equation*}
bestimmt, sodass vermutet wird, dass es sich um einen Würfel aus Holz handelt, welches einen Absorptionskoeffizienten im Bereich von $\mu_{\text{Holz}} \in
[\SI[per-mode=reciprocal]{0.024}{\per\centi\metre}\, , \SI[per-mode=reciprocal]{0.096}{\per\centi\metre}]$ hat. Damit ergibt sich eine Abweichung zur oberen Grenze von 
\begin{equation*}
    \increment \mu_{\text{Würfel 2}} = \SI{9(7)}{\percent}\, ,
\end{equation*}
welche über die Formel $\increment x = 1 - \frac{x_{\text{exp}}}{x_{\text{lit}}}$ berechnet wird. \\
Für den Absorptionskoeffizienten von Würfel 3 wird 
\begin{equation*}
    \mu_{\text{Würfel 3}} = \SI{1.358(71)}{\per\centi\metre}
\end{equation*}
ermittelt. Es wird vermutet, dass es sich um einen Würfel aus Blei handelt, $\mu_{\text{Blei}} = \SI[per-mode=reciprocal]{1.415}{\per\centi\metre}$. Der gemessene Wert 
weicht um 
\begin{equation*}
    \increment \mu_{\text{Würfel 3}} = \SI{4(5)}{\percent}
\end{equation*}
von dem Literaturwert ab.\\
Der Würfel 4 besteht aus 27 verschiedenen kleineren Würfeln, da nur eine Ebene untersucht wird, werden 9 dieser Würfel bestimmt. Der experimentell ermittelten 
Wert ist neben dem vermuteten Stoff und der relativen Abweichung in der \autoref{tab:compare} aufgelistet. Da bei Holz die Dichte $\rho$ ja nach verwendeter Holzart
und Feuchte des Holzes stark unterschiedliche Werte annimmt, ist es schwierig einen festen Literaturwert anzugeben. In der \autoref{tab:lit} wird dies über die 
Angabe eines Bereiches umgangen. Hier werden nun keine Theoriewerte und keine Abweichungen bestimmt, wenn der ermittelte Wert in dem angebenen Bereich liegt. 
Der Wert des Absoprtionskoeffizienten, der für den Würfel 7 ermittelt wurde, liegt leicht oberhalb des Bereiches, sodass eine Abweichung von der oberen Grenze 
berechnet wird. 

\begin{table}[H]
    \centering
    \caption{Die ermittelten Werte für die Absorptionskoeffizienten der verschiedenen kleineren Würfel neben dem vermuteten Stoff und der jeweiligen Abweichung.}
    \label{tab:compare}
    \begin{tabular}{c S[table-format=1.3] @{${}\pm{}$} S[table-format=1.3] S[table-format=1.3] c S[table-format=2.1] @{${}\pm{}$} S[table-format=2.1]}
      \toprule
      {Würfel} & \multicolumn{2}{c}{$\mu \mathbin{/} \si{\per\centi\metre}$}  & {$\mu_{\text{lit}} \mathbin{/} \si{\per\centi\metre}$} & {Stoff} & \multicolumn{2}{c}{$\increment\mu \mathbin{/} \si{\percent}$} \\
      \midrule
      1 & 0.238 & 0.024  & 0.211 & Aluminium  & 13    &11\\
      2 & 1.148 & 0.023  & 1.415 & Blei       & 18.8  & 1.7\\
      3 & 0.276 & 0.024  & 0.211 & Aluminium  & 31    &12\\
      4 & 0.080 & 0.020  &       & Holz       &       & \\
      5 & 1.154 & 0.025  & 1.415 & Blei       & 18.4  &1.8\\
      6 & 0.089 & 0.020  &       & Holz       &      &  \\
      7 & 0.100 & 0.023  & 0.096 & Holz       & 5    & 25\\
      8 & 1.063 & 0.023  & 1.415 & Blei       & 24.9  &1.7\\
      9 & 0.045 & 0.023  &       & Holz       &       &  \\
      \bottomrule
    \end{tabular}
  \end{table}

\noindent Es ist zu sehen, dass die Zuordnung der Materialen bei Würfel 4 deutlich weniger genau ist als bei den homogenen Würfeln 2 und 3. Ein Grund
dafür ist, dass sich der Strahl hinter der Lochblende der Quelle kegelförmig ausbreitet, und so nicht nur durch die gewählten Würfel geht. 
Besonders bei den Diagonalmessungen ist dieser Effekt groß. Dazu kommt die Einstellung des Strahlengangs per Augenmaß. Es wird also auch in anderen, 
bei einer Projektion nicht beachteten Würfeln mitabsorbiert, sodass die Messung deutlich verfälscht wird. Es hätten natürlich andere Projektionen gewählt 
werden können, welche die Auswertung aber mathematisch verkomplizieren. 